Current and future surveys of the Galaxy contain a wealth of
information about the structure and evolution of the Galactic disk and
halo. Teasing out this information is complicated by measurement
uncertainties, missing data, and sparse sampling. I describe several
applications of generative modeling--—modeling the data by specifying
and constraining how the data were generated--—to deal with these
issues. I infer the velocity distribution of nearby stars from
\hipparcos\ data, which only consists of tangential velocities, by
forward modeling. I characterize the contribution of several ``moving
groups''---overdensities of co-moving stars---to the full
distribution. By studying the properties of stars in these moving
groups, I show that they do not form a single-burst population and
that they are most likely due to transient non-axisymmetric features
of the disk. By forward modeling one such scenario, I show how the
Hercules moving group can be traced around the Galaxy by future
surveys. I develop a method for inferring the Galactic potential from
individual observations of stellar kinematics. This method takes
uncertainties in our knowledge of the distribution function into
account through marginalization. I demonstrate the method by inferring
the force law in the Solar System from observations of the positions
and velocities of the eight planets at a single epoch. I apply a
similar method to derive the Milky Way's circular velocity from
observations of maser kinematics.
