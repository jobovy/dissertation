Current and future surveys of the Galaxy contain a wealth of
information about the structure and evolution of the Galactic disk and
halo. Teasing out this information is complicated by measurement
uncertainties, missing data, and sparse sampling. I develop and
describe several applications of generative modeling--—creating an
approximate description of the probability of the data given the
physical parameters of the system--—to deal with these issues.

I develop a method for inferring the Galactic potential from
individual observations of stellar kinematics such as will be
furnished by the upcoming \Gaia\ space astrometry mission. This method
takes uncertainties in our knowledge of the distribution function of
stellar tracers into account through marginalization. I demonstrate
the method by inferring the force law in the Solar System from
observations of the positions and velocities of the eight planets at a
single epoch. I apply a similar method to derive the Milky Way's
circular velocity from observations of maser kinematics.

I infer the velocity distribution of nearby stars from
\hipparcos\ data, which only consist of tangential velocities, by
forward modeling the underlying distribution with a flexible
multi-Gaussian model. I characterize the contribution of several
``moving groups''---overdensities of co-moving stars---to the full
distribution. By studying the properties of stars in these moving
groups, I show that they do not form a single-burst population and
that they are most likely due to transient non-axisymmetric features
of the disk, such as transient spiral structure. By forward modeling
one such scenario, I show how the Hercules moving group can be traced
around the Galaxy by future surveys, which would confirm that the
Milky Way bar's outer Lindblad resonance lies near the Solar radius.
