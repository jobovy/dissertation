Discuss ISM. 

This dissertation has addressed several of the fundamental questions
concerning our Galaxy raised in the introduction. We have made
significant progress in the methodological questions as to how we can
infer the gravitational potential from kinematic data, such as is the
problem \Gaia\ faces. Many questions remain, especially concerning the
feasibility of the approach on the full \Gaia\ set or even for any
informative tracer population. The detailed modeling of the
non-axisymmetric Galaxy and determining whether our fits actually
represent a steady state using $N$-body simulations will be another
thing to fold in, and the best way to do this is not clear. The
optimal way to do this will probably consist of importance sampling
simple steady-state fits using detailed $N$-body calculations.

We have learned several facts about the evolution of the disk and the
potential of our Galaxy. We inferred the Milky Way's circular velocity
from a set of maser observations. By studying the local velocity
distribution we were able to constrain the non-axisymmetric structures
in the Galaxy convolved with the chemical evolution of the disk. We
showed that there must be a transient component to the Galactic bar
and spiral structure to explain the observations of the local
kinematics. We also showed how future observations can further
constrain these scenarios.
