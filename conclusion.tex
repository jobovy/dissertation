This dissertation has addressed several of the questions concerning
our Galaxy raised in the introduction. We have shown that the
dynamical inference problem of finding the Galaxy's gravitational
potential, and thus its density profile, from observational kinematic
data is possible in \chaptername s~\ref{chap:solarsystem}
and \ref{chap:masers}. Compared to the full \Gaia\ data set of 10$^9$
objects, the problems considered in these chapters are much smaller (8
planets in \chaptername~\ref{chap:solarsystem}; 18 masers
in \chaptername~\ref{chap:masers}). It will take considerable effort
to adapt the method introduced in \chaptername~\ref{chap:solarsystem}
to much larger data sets without making more assumptions about the
orbital distribution of the tracer set. An additional problem that
will crop up when analyzing the full \Gaia\ data set is that there
will be significant contaminating populations, such as streams of
stars that are in the process of being accreted by the Milky
Way. Since these contaminants do not constitute a relaxed,
steady-state population, they have the potential of biasing the
dynamical inference, and the data analysis methods employed will need
to deal with these contaminants. We note, however, that such streams
could be very informative about the Galactic potential in themselves,
using the constraint that they should be consistent with being
composed of stars stripped from a parent
object \citep[\eg,][]{Koposov09a}.

Intermediate-scale problems exist that can bridge the gap between the
small contamination-free data sets used in this dissertation and the
full \Gaia\ data set. These are, for example, the nuclear star cluster
in the Milky Way, where kinematic data is available for thousands of
stars within the central 1 pc \citep{Schoedel09a} that can be used to
infer the gravitational potential near the Milky Way's central black
hole; detailed kinematic data on globular clusters that can be used to
infer the clusters mass profile, including the contribution from an
intermediate-mass black hole \citep{Anderson10a}; and the local Solar
neighborhood, where a wealth of data exists that can be used to infer
the local vertical---\ie, perpendicular to the plane of the
disk---potential \citep[\eg, RAVE;][]{Zwitter08a}. The first two of
these problems are qualitatively similar to the inference problem
considered in \chaptername~\ref{chap:solarsystem} in that these
systems are roughly spherically symmetric and almost certainly
relaxed. The vertical-potential problem is a nice test problem
for \Gaia\ in that it will have similar contamination of non-relaxed
populations as \Gaia; \Gaia\ will in fact be able to perform almost
exactly the same measurement at other locations in the Galaxy, which
will help pin down the disk's surface-mass density profile and, thus,
the contribution of the dark halo to the inner Milky Way potential.

We have ignored additional information for the dynamical inference
that comes from elemental abundance information. \Gaia\ will not
obtain detailed elemental abundances, so these data will be sparse
compared to the full \Gaia\ data set. Nevertheless, sparse
elemental-abundance data for a small set of tracers could be very
informative because it will help in separating different components of
the Galaxy, for example, to separate non-relaxed contaminating
populations from the relaxed population used in the inference.


We have learned several facts about the evolution of the Galactic
disk. By studying the local velocity distribution
(\chaptername~\ref{chap:veldist}) we were able to constrain the
non-axisymmetric structures in the Galaxy convolved with the chemical
evolution of the disk (\chaptername~\ref{chap:groups}). We showed that
there must be a transient component to the Galactic bar and spiral
structure to explain the observations of the local
kinematics. However, the absence of global data for the Milky Way's
disk and the absence of more detailed elemental-abundance data means
that these constraints are weak and qualitative at best. As described
in the Introduction, \apogee\ will survey the disk globally with
detailed elemental-abundance data for $\approx 100,000$ giant stars
(starting in May 2011). These data will allow the non-axisymmetric
components of the disk to be constrained in detail. We have shown for
one particular scenario---the bar-origin model of the Hercules moving
group---how \apogee, or a similar survey, could test this hypothesis
outside of the immediate Solar neighborhood and constrain the Galactic
bar. Similar predictions will be made for other scenarios such that
they can be tested using \apogee. These predictions can be purely
kinematic (as in \chaptername~\ref{chap:hercules}), but in combination
with chemical evolution models, they can also specify the
distributions of elemental abundances across the disk. In the end a
full chemo-dynamical model of the Galactic disk should be constructed
and tested against the \apogee\ data.

One thing that has been conspicuously absent in this dissertation is
the interstellar medium; except for the masers we have focused mostly
on non-collisional tracers. The gaseous disk of the Milky Way only
contributes about 10\,percent of the disk's mass and is therefore not
that dynamically relevant (because it is mostly axisymmetric, it does
not give rise to strong dynamical effects near orbital resonances, as
is the case for the percent-level bar and spiral contributions to the
disk mass). Nevertheless, the gas in the disk does play an important
dynamical role in some important cases: the reservoir of cold gas from
which stars form is the fuel that keeps spiral structure running in
galactic disks (without the gas the spirals would dissolve themselves
through spiral heating; \citealt{Carlberg85a}; because gas has to
orbit on non-intersecting orbits to avoid shocks, the absence of
strong shocks is a constraint on the orbital space and thus on the
potential, which needs to permit non-intersecting orbits. For
dynamical inference gas can also be very informative: because it
prefers close-to circular, non-intersecting orbits, measuring the
kinematics of the gas gives a direct measurement of the Milky Way's
circular velocity at different radii; like stars, gas can pile up at
orbital resonances of the bar and thus be used to constrain the
bar. Even though upcoming surveys are focusing on stellar tracers,
further investigations of the dynamical role of the interstellar
medium are warranted.
