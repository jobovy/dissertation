%% NYU PhD thesis format. Created by Jos� Koiller 2007--2008.

%% Use the first of the following lines during production to
%% easily spot "overfull boxes" in the output. Use the second
%% line for the final version.
%\documentclass[12pt,draft,letterpaper]{report}
\documentclass[12pt,letterpaper]{report}

%% Replace the title, name, advisor name, graduation date and dedication below with
%% your own. Graduation months must be January, May or September.
\newcommand{\thesistitle}{TITLE}
\newcommand{\thesisauthor}{Jo~Bovy}
\newcommand{\thesisadvisor}{Professor David~W.~Hogg}
\newcommand{\graddate}{May 2011}
%% If you do not want a dedication, scroll down and comment out
%% the appropriate lines in this file.
%\newcommand{\thesisdedication}{To my dog Weierstra\ss, with affection.}

%% The following makes chapters and sections, but not subsections,
%% appear in the TOC (table of contents). Increase to 2 or 3 to
%% make subsections or subsubsections appear, respectively. It seems
%% to be usual to use the "1" setting, however.
\setcounter{tocdepth}{1}

%% Sectional units up to subsubsections are numbered. To number
%% subsections, but not subsubsections, decrease this counter to 2.
\setcounter{secnumdepth}{3}

%% Page layout (customized to letter paper and NYU requirements):
\setlength{\oddsidemargin}{.6in}
\setlength{\textwidth}{5.8in}
\setlength{\topmargin}{.1in}
\setlength{\headheight}{0in}
\setlength{\headsep}{0in}
\setlength{\textheight}{8.3in}
\setlength{\footskip}{.5in}

%% Use the following commands, if desired, during production.
%% Comment them out for final version.
\usepackage{layout} % defines the \layout command, see below
%\setlength{\hoffset}{-.75in} % creates a large right margin for notes and \showlabels

%% Controls spacing between lines (\doublespacing, \onehalfspacing, etc.):
\usepackage{setspace}

%% Use the line below for official NYU version, which requires
%% double line spacing. For all other uses, this is unnecessary,
%% so the line can be commented out.
%\doublespacing % requires package setspace, invoked above

%% Each of the following lines defines the \com command, which produces
%% a comment (notes for yourself, for instance) in the output file.
%% Example:    \com{this will appear as a comment in the output}
%% Choose (uncomment) only one of the three forms:
%\newcommand{\com}[1]{[/// {#1} ///]}       % between [/// and ///].
\newcommand{\com}[1]{\marginpar{\tiny #1}} % as (tiny) margin notes
%\newcommand{\com}[1]{}                     % suppress all comments.

%% This inputs your auxiliary file with \usepackage's and \newcommand's:
%% It is assumed that that file is called "definitions.tex".
% Graphics:
\usepackage[final]{graphicx}
%\usepackage{graphicx} % use this line instead of the above to suppress graphics in draft copies
%\usepackage{graphpap} % \defines the \graphpaper command

%%BIBSTYLE
\usepackage[authoryear]{natbib}
\bibpunct{(}{)}{;}{a}{}{,}

% Indent first line of each section:
\usepackage{indentfirst}

% Fonts and symbols:
\usepackage{amsfonts,amssymb,amsmath}

%% others
\usepackage{url}
\usepackage{aas_macros}
\usepackage{deluxetable}

%%newcommands
%%Latin
\newcommand{\etal}{et~al.}
\newcommand{\eg}{e.g.}
\newcommand{\ie}{i.e.}

%%Figures etc
\newcommand{\figurenames}{\figurename s}
\newcommand{\sectionname}{\S}
\newcommand{\eqnname}{equation}


%%Solar system paper
\newcommand{\unit}[1]{\mathrm{#1}}
\newcommand{\AU}{\unit{AU}}
\newcommand{\unitday}{\unit{d}}
\newcommand{\yr}{\unit{yr}}
\newcommand{\satellite}[1]{\textsl{#1}}
\newcommand{\Gaia}{\satellite{Gaia}}
\newcommand{\KS}{K--S}
\newcommand{\tvector}[1]{\boldsymbol{\vec{#1}}}
\newcommand{\vx}{\tvector{x}}
\newcommand{\vv}{\tvector{v}}
\newcommand{\vxi}{\tvector{x}_i}
\newcommand{\vvi}{\tvector{v}_i}
\newcommand{\va}{\tvector{a}}
\newcommand{\vI}{\tvector{I}}
\newcommand{\vphi}{\tvector{\phi}}
\newcommand{\tuvector}[1]{\boldsymbol{\hat{#1}}}
\newcommand{\rhat}{\tuvector{r}}
\newcommand{\mvector}[1]{\boldsymbol{{#1}}}
\newcommand{\mvtheta}{\mvector{\theta}}
\newcommand{\mvthetae}{\mvector{\theta}_e}
\newcommand{\mvthetaepsilon}{\mvector{\theta}_{\epsilon}}
\newcommand{\mvomega}{\mvector{\omega}}
\newcommand{\mvomegahatk}{\mvector{\hat{\omega}}_k}
\newcommand{\mvDelta}{\mvector{\Delta}}
\newcommand{\dd}{\mathrm{d}}
\newcommand{\setofxv}{\{\vx_i,\vv_i\}}
\newcommand{\setofrv}{\{r_i,v_{r,i}\}}
\newcommand{\ueff}{u_{\mathrm{eff}}}
\newcommand{\rperi}{r_{\mathrm{peri}}}
\newcommand{\rap}{r_{\mathrm{ap}}}
\renewcommand{\angle}{\phi_r}
\newcommand{\anglei}{\phi_{r,i}}
\newcommand{\ff}{f_{\mvtheta}}
\newcommand{\fff}{\tilde{\ff}}
\newcommand{\lnepsilon}{\ln{\epsilon}}
\newcommand{\vperp}{j^2} % note strange definition
\newcommand{\vperpi}{j_i^2}
\newcommand{\gradient}{\mvector{\nabla}_{\!\!\mvomega}}
\newcommand{\wk}{\ensuremath{w_k}}
\newcommand{\nk}{\ensuremath{n_k}}



%%MASERS
\newcommand{\lsr}{LSR}
\newcommand{\vsunlsr}{\ensuremath{v_\odot}}
\newcommand{\vsunlsrx}{\ensuremath{v_{\odot,x}}}
\newcommand{\vsunlsry}{\ensuremath{v_{\odot,y}}}
\newcommand{\vsunlsrz}{\ensuremath{v_{\odot,z}}}
\newcommand{\vlsr}{\ensuremath{v_{\mbox{\footnotesize \lsr}}}}
\newcommand{\omegao}{\ensuremath{\Omega_0}}
\newcommand{\Ro}{\ensuremath{R_0}}
\newcommand{\zo}{\ensuremath{z_0}}
\renewcommand{\vec}[1]{\mathbf{#1}} % boldface for vectors
\newcommand{\mm}{\ensuremath{\vec{m}}}
\newcommand{\zerovector}{\ensuremath{\vec{0}}}
\newcommand{\masermean}{\ensuremath{\overline{\vv}}}
\newcommand{\samplemean}{\ensuremath{\overline{\tvector{\mu}}}}
\newcommand{\mmR}{\ensuremath{\overline{v}_R}}
\newcommand{\mmphi}{\ensuremath{\overline{v}_\phi}}
\newcommand{\mmz}{\ensuremath{\overline{v}_z}}
\newcommand{\vvpec}{\ensuremath{\vv_{\mbox{{\footnotesize pec}}}}}
\newcommand{\vpec}{\ensuremath{v_{\mbox{{\footnotesize pec}}}}}
\newcommand{\vvmaser}{\ensuremath{\vv_{\mbox{{\footnotesize maser}}}}}
\newcommand{\eeR}{\ensuremath{\vec{e}_{R}}}
\newcommand{\eephi}{\ensuremath{\vec{e}_{\phi}}}
\newcommand{\eez}{\ensuremath{\vec{e}_{z}}}
\newcommand{\vvRi}{\ensuremath{\vpec_{R,i}}}
\newcommand{\vvphii}{\ensuremath{\vpec_{\phi,i}}}
\newcommand{\vvzi}{\ensuremath{\vpec_{z,i}}}
\newcommand{\vvpeci}{\ensuremath{\vvpec_i}}
\newcommand{\ten}[1]{\mathbf{#1}} % boldface for tensors
\newcommand{\RR}{\ten{R}}
\newcommand{\VV}{\ten{V}}
\newcommand{\WW}{\ten{W}}
\newcommand{\II}{\ten{I}}
\newcommand{\TT}{\ten{T}}
\newcommand{\AAA}{\ten{A}}
\newcommand{\maserdisp}{\ensuremath{\mbox{\boldmath$\sigma$}}}
\newcommand{\vgalx}{\ensuremath{\tilde{v}_x}}
\newcommand{\vgaly}{\ensuremath{\tilde{v}_y}}
\newcommand{\vgalz}{\ensuremath{\tilde{v}_z}}
\newcommand{\normal}{\ensuremath{\mathcal{N}}}
\newcommand{\wishart}{\ensuremath{\mathcal{W}}}
\newcommand{\gammadist}{\ensuremath{\mathcal{G}}}
\newcommand{\trace}{\mbox{Trace}}
\newcommand{\T}{^{\scriptscriptstyle \top}}   % transpose
\newcommand{\offsets}{\ensuremath{\Delta\vx_i, \Delta \vv_i}}
\newcommand{\alloffsets}{\ensuremath{\{\offsets\}}}
\newcommand{\pmsgra}{\ensuremath{\mu_{\mbox{{\footnotesize Sgr A$^*$}}}}}
\newcommand{\vres}{\ensuremath{\tvector{v}_{\mbox{{\footnotesize res}}}}}
\newcommand{\vresi}{\ensuremath{\tvector{v}_{\mbox{{\footnotesize res}}, i}}}
\newcommand{\vgal}{\ensuremath{\tvector{v}_{\mbox{{\footnotesize Gal}}}}}
\newcommand{\vgali}{\ensuremath{\tvector{v}_{\mbox{{\footnotesize Gal}}, i}}}

\newcommand{\vlba}{VLBA}
\newcommand{\vlbi}{VLBI}
\newcommand{\vera}{VERA}
\newcommand{\reid}{R09}
\newcommand{\vc}{V_c}
\newcommand{\mw}{MW}

\newcommand{\ra}{\ensuremath{\alpha}}
\newcommand{\dec}{\ensuremath{\delta}}
\newcommand{\pmra}{\ensuremath{\mu_{\ra}}}
\newcommand{\pmdec}{\ensuremath{\mu_{\dec}}}
\newcommand{\eqx}{\ensuremath{x_{\textnormal{eq}}}}
\newcommand{\eqy}{\ensuremath{y_{\textnormal{eq}}}}
\newcommand{\eqz}{\ensuremath{z_{\textnormal{eq}}}}
\newcommand{\gall}{{\it l}}
\newcommand{\galb}{{\it b}}
\newcommand{\pmll}{\ensuremath{\mu_\gall}}
\newcommand{\pmbb}{\ensuremath{\mu_\galb}}
\newcommand{\galx}{\ensuremath{x}}
\newcommand{\galy}{\ensuremath{y}}
\newcommand{\galz}{\ensuremath{z}}
\newcommand{\galU}{\ensuremath{U}}
\newcommand{\galV}{\ensuremath{V}}
\newcommand{\galW}{\ensuremath{W}}
\newcommand{\parallax}{\ensuremath{\varpi}}
\newcommand{\radialdist}{\ensuremath{d}}
\newcommand{\vrr}{\ensuremath{v_r}}
\newcommand{\vll}{\ensuremath{v_\gall}}
\newcommand{\vbb}{\ensuremath{v_\galb}}
\newcommand{\vernal}{\ensuremath{\Upsilon}}
\newcommand{\ngp}{\textnormal{NGP}}
\newcommand{\ngpg}{\textnormal{G}}
\newcommand{\ncp}{\textnormal{NCP}}
\newcommand{\ncpp}{\textnormal{P}}
\newcommand{\gc}{\textnormal{GC}}
\newcommand{\gcc}{\textnormal{C}}
\newcommand{\rangp}{\ensuremath{\ra_\ngp}}
\newcommand{\decngp}{\ensuremath{\dec_\ngp}}
\newcommand{\ragc}{\ensuremath{\ra_\gc}}
\newcommand{\decgc}{\ensuremath{\dec_\gc}}
\newcommand{\degree}{^{\circ}}
\newcommand{\matrixleft}{\left[}
\newcommand{\matrixright}{\right]}
\newcommand{\arcsecs}{\textnormal{as}}

%%HERCULES
\newcommand{\apogee}{APOGEE}
\newcommand{\hermes}{HERMES}
\newcommand{\hipparcos}{\emph{Hipparcos}}
\newcommand{\vR}{\ensuremath{v_R}}
\newcommand{\vphihercules}{\ensuremath{v_{\phi}}}
\newcommand{\vo}{\ensuremath{v_0}}
\newcommand{\ro}{\Ro}
\newcommand{\Ab}{\ensuremath{A_b}}
\newcommand{\Rb}{\ensuremath{R_b}}
\newcommand{\Omegab}{\ensuremath{\Omega_b}}
\newcommand{\fdehnen}{\ensuremath{f_{\text{Dehnen}}}}
\newcommand{\sigmaR}{\ensuremath{\sigma_R}}
\newcommand{\rE}{\ensuremath{R_e}}
\newcommand{\Lc}{\ensuremath{L_c}}
\newcommand{\Rs}{\ensuremath{R_s}}
\newcommand{\Rsigma}{\ensuremath{R_{\sigma}}}
\newcommand{\Rolr}{\ensuremath{R_{\text{OLR}}}}


%% Cross-referencing utilities. Use one or the other--whichever you prefer--
%% but comment out both lines for final version.
%\usepackage{showlabels}
%\usepackage{showkeys}


\begin{document}
%% Produces a test "layout" page, for "debugging" purposes only.
%% Comment out for final version.
\layout % requires package layout (see above, on this same file)

%%%%%% Title page %%%%%%%%%%%
%% Sets page numbering to "roman style" i, ii, iii, iv, etc:
\pagenumbering{roman}
%
%% No numbering in the title page:
\thispagestyle{empty}
%
\begin{center}
  {\large\textbf{\thesistitle}}
  \vspace{.7in}

  by
  \vspace{.7in}

  \thesisauthor
  \vfill

\begin{doublespace}
  A dissertation submitted in partial fulfillment\\
  of the requirements for the degree of\\
  Doctor of Philosophy\\
  Department of Physics\\
  New York University\\
  \graddate
\end{doublespace}
\end{center}
\vfill

\noindent\makebox[\textwidth]{\hfill\makebox[2.5in]{\hrulefill}}\\
\makebox[\textwidth]{\hfill\makebox[2.5in]{\hfill\thesisadvisor\hfill}}
\newpage
%%%%%%%%%%%%% Blank page %%%%%%%%%%%%%%%%%%
\thispagestyle{empty}
\vspace*{0in}
\newpage

%%%%%%%%%%%%%% Dedication %%%%%%%%%%%%%%%%%
%% Comment out the following lines if you do not want to dedicate
%% this to anyone...
%\vspace*{\fill}
%\begin{center}
%  \thesisdedication\addcontentsline{toc}{section}{Dedication}
%\end{center}
%\vfill
%\newpage
%%%%%%%%%%%%%% Acknowledgements %%%%%%%%%%%%
%% Comment out the following lines if you do not want to acknowledge
%% anyone's help...
\section*{Acknowledgements}\addcontentsline{toc}{section}{Acknowledgements}


Hogg

Fr{\'e}d{\'e}ric Arenou, Michael Aumer, Coryn Bailer-Jones, James
Binney, Michael Blanton, Anthony Brown, Daniela Carollo, Ilias Cholis,
Kyle Cranmer, Walter Dehnen, Mulin Ding, Glennys Farrar, Ken Freeman,
Stefan Gillessen, Andrei Gruzinov, Kathryn Johnston, Joe Hennawi, Matt
Kleban, Dustin Lang, Floor van Leeuwen, Yuri Levin, Stephen Levine,
Tom Loredo, Dmitry Malyshev, Phil Marshall, Surhud More, John
Moustakas, Iain Murray, Adam Myers, Bill Press, Mark Reid, Hans-Walter
Rix, Sam Roweis, Erin Sheldon, Michael Strauss, Scott Tremaine, Glenn
van de Ven, David Weinberg, Neal Weiner, and a few anonymous referees.

Funding agencies: the National Aeronautics and Space Administration
(ADP grant NNX08AJ48G) and the National Science Foundation (grant
AST-0908357). Partial support from the Max-Planck-Institut f\"ur
Astronomie, a New York University Horizon fellowship, and a Horizon
Dissertation writing fellowship. I would also like to acknowledge the
hospitality of The Max-Planck-Institut f\"ur Astronomie and the
Lorentz Center in Leiden.

Software: the HORIZONS System provided by the Solar System Dynamics
Group of the Jet Propulsion Laboratory, the NASA Astrophysics Data
System, and the open-source Python modules scipy, numpy, and
matplotlib.




\newpage
%%%% Abstract %%%%%%%%%%%%%%%%%%
\section*{Abstract}\addcontentsline{toc}{section}{Abstract}
Current and future surveys of the Galaxy contain a wealth of
information about the structure and evolution of the Galactic disk and
halo. Teasing out this information is complicated by measurement
uncertainties, missing data, and sparse sampling. I develop and
describe several applications of generative modeling--—creating an
approximate description of the probability of the data given the
physical parameters of the system--—to deal with these issues.

I develop a method for inferring the Galactic potential from
individual observations of stellar kinematics such as will be
furnished by the upcoming \Gaia\ space astrometry mission. This method
takes uncertainties in our knowledge of the distribution function of
stellar tracers into account through marginalization. I demonstrate
the method by inferring the force law in the Solar System from
observations of the positions and velocities of the eight planets at a
single epoch. I apply a similar method to derive the Milky Way's
circular velocity from observations of maser kinematics.

I infer the velocity distribution of nearby stars from
\hipparcos\ data, which only consist of tangential velocities, by
forward modeling the underlying distribution with a flexible
multi-Gaussian model. I characterize the contribution of several
``moving groups''---overdensities of co-moving stars---to the full
distribution. By studying the properties of stars in these moving
groups, I show that they do not form a single-burst population and
that they are most likely due to transient non-axisymmetric features
of the disk, such as transient spiral structure. By forward modeling
one such scenario, I show how the Hercules moving group can be traced
around the Galaxy by future surveys, which would confirm that the
Milky Way bar's outer Lindblad resonance lies near the Solar radius.

\newpage
%%%% Table of Contents %%%%%%%%%%%%
\tableofcontents

%%%%% List of Figures %%%%%%%%%%%%%
%% Comment out the following two lines if your thesis does not
%% contain any figures. The list of figures contains only
%% those figures included withing the "figure" environment.
\listoffigures\addcontentsline{toc}{section}{List of Figures}
\newpage

%%%%% List of Tables %%%%%%%%%%%%%
%% Comment out the following two lines if your thesis does not
%% contain any tables. The list of tables contains only
%% those tables included withing the "table" environment.
\listoftables\addcontentsline{toc}{section}{List of Tables}
\newpage

%%%%% Body of thesis starts %%%%%%%%%%%%
\pagenumbering{arabic} % switches page numbering to arabic: 1, 2, 3, etc.
%% Introduction. If your thesis has no introduction, or chapter 1 is
%% meant to be the introduction, then comment out the lines below.
\section*{Introduction}\addcontentsline{toc}{section}{Introduction}
In the current understanding of the formation of structure in the
Universe the initial tiny density fluctuations are generated during an
inflationary epoch. The dominant mass component---the dark
matter---starts to cluster under the influence of gravity after the
dark matter decouples from the other constituents of the early
Universe, while the ordinary matter of atoms and everyday life remains
coupled to radiation until about 300,000 years after the Big
Bang. When the temperature of the Universe drops below about 3,000 K,
ordinary matter decouples from radiation and falls into the potential
wells created by dark-matter clustering. Ordinary matter is
predominantly found as neutral hydrogen at this stage and this gas
settles at high density at the bottoms of the dark matter
potentials. Stars and later galaxies form in these high density regions.

The current paradigm has been successful in explaining the large-scale
properties of the Universe, especially the close-to-linear regime of
the Cosmic Microwave Background, formed at the epoch of
ordinary-matter--radiation decoupling, and the largest ($> 100 Mpc$
scales in the nearby Universe, where the non-linear influence of
gravity does not play a large role. However, the formation and
evolution of individual galaxies remains to be understood. This is
obvious from the many classes of galaxies and other objects identified
locally (\eg, the Hubble sequence, dwarf Spheroidal galaxies, dwarf
irregular galaxies, etc.) and at higher redshift (\eg, Ultraluminous
infrared galaxies, sub-millimeter galaxies, Lyman-limit systems,
Damped Lyman-$\alpha$ absorbers, MgII absorbers, etc.). How all these
systems form and relate to one another is yet to be explained. 

Since most of Galactic and extra-galactic astronomy is limited to
snapshots at a particular time, one can study the formation and
evolution of galaxies either by looking at galaxies at different
redshifts and interpreting these observations as different stages in
the evolution of present-day galaxies, or one can study galaxies
locally, where observations with much more detail are possible,
especially for the Milky Way and M31, and infer their formation and
evolution. These two approaches move in remarkable lockstep, as the
highest redshifts probed (BOVY: CHANGE) by current and planned surveys
are comparable to the times at which the lowest metallicity stars
currently studied in the Milky Way formed (BOVY: CITE
BLAND+FREEMAN). These approaches are therefore strongly
interrelated. This dissertation focuses on the latter approach.



Absence of major merger for MW (10 Gyr, redshift ?). 


It is known for certain that there is a large amount of dark mass
beyond the Sun’s position in the Milky Way. We expect there to be dark
matter in the inner region of the Galaxy as well. The determination of
the mass distribution in the inner Galaxy, and its decomposition into
the stellar and dark components, is made difficult by the ambiguity of
local stars as tracers of the mass distribution; certain distracting
disk dynamical features such as spiral arms; and our own (moving)
position in the middle of all of this.

While the Milky Way is interesting in its own right—--it is our home
in the Universe--—it is only one among thousands of galaxies in the
local Universe. Despite this, the Milky Way is one of the main sites
for studying structure formation in the Universe on small scales,
because of the amazing amount of detail possible in these
observations. For comparison, the apparent size of the Milky Way’s
nearest neighbor, the Andromeda galaxy, is only a few times that of
the full moon; other nearby galaxies are much farther away. Since we
have no reason to believe that the Milky Way is atypical in any
way—--we can detect many galaxies with similar features in the night
sky—the details of the Milky Way’s mass distribution and formation
history also provide a general understanding of how galaxies form.


Since stars move across the sky at a slow pace and because they orbit
the Galaxy on timescales measured in hundreds of millions of years, we
can hope at best to measure their positions and velocities at the
present time, while their accelerations are beyond our observational
reach. This poses a fundamental difficulty for interpreting the data
in terms of the mass distribution that guides them.  Newton’s second
law says that the acceleration of a star is directly set by the force
acting on it, while its velocity merely reflects the initial motion of
the object and not the underlying force. Thus, without the
accelerations of stars it seems that we cannot know the underlying
mass distribution. This problem and its resolution is discussed in
Chapter I.

To derive the mass distribution of the Milky Way from stellar motions,
we need to make additional assumptions about the kinematics of stars
that couple these motions to the mass distribution. One such
assumption is that the distribution of the stellar positions and
velocities is time-independent.  This seemingly simple assumption
provides the coupling between the underlying mass distribution and the
stellar kinematics. As an example, I show that we can determine the
force law in the Solar System from observations of the planets’
positions and velocities at one epoch. The nontrivial part of this toy
data set’s analysis is that one has to model the distribution of the
planets’ positions and velocities in a timeindependent manner while
not introducing extra assumptions. I achieve this by building a full
probabilistic model of the data that introduces extra degrees of
freedom describing the time-independent distribution of the positions
and velocities; these extra degrees of freedom can be integrated over
to obtain the final result for the force law in the Solar System which
is independent of the extra degrees of freedom.

Another application considers not observations of stars directly, but
the light of young stars, reprocessed and amplified by the molecular
gas clouds surrounding them. These masers—--lasers on a cosmic
scale—--can be seen to large distances, and their kinematics can be
measured by radio observatories much more precisely than is possible
for stars with optical telescopes. The current data set of 18 masers
contains much information about the disk’s structure, but the
interpretation of the data again depends on our assumptions about the
distribution of their positions and velocities. By inferring a simple
model in which the velocities of the masers are scattered around a
mean offset from the velocity associated with circular motion, we
obtain this circular velocity at the Sun’s position in the
Galaxy. This tells us directly about the Galaxy’s mass within the
Solar radius. This work has been published (Bovy et al. 2009,
Astrophys. J.  704, 1704). 

Chapter III concerns the available data on the motions of stars in the
Galactic disk. The Hipparcos satellite has accurately measured the
positions and motions of stars close to the Sun, which allows us to
look at their detailed properties. First I reconstruct the local
velocity distribution from these data. The result is complex which one
would not expect if the assumptions—axisymmetry and timeindependence
—introduced in the previous paragraphs were correct (this work has
been published: Bovy et al. 2009, Astrophys. J. 700, 1794). Looking at
the mixture of stars that make up the unexpected complexity, I find
that these stars have not formed together such that the complexity is
not due to a formation-history time-dependence and axisymmetry must be
broken. By considering how consistently we can measure the Sun’s
motion relative to the local circular velocity (measured in chapter
II)—--Assuming axisymmetry, this measurement is possible—--I find that
the disk is non-axisymmetric at the level of a few percent.

%% If your thesis has different "Parts", use commands such as the following:
%\part{First Part\label{part:one}}%
%\input{chap1}
%\input{chap2} % further chapters -- change file names to meaningful things...
%\input{chap3}
%\part{Second Part\label{part:two}}%
%\input{chap4}
%\input{chap5}
%\input{chap6}
%%%%% Appendices start %%%%%%%%%%%%%%%%
%% Comment out the following line if your thesis has no appendix
%\appendix
%\input{app1}
%% Note: If your thesis has more than one appendix, NYU requires a "list of
%% appendices" page before the body of the thesis. I don't provide the tools
%% to create that here, so you're on your own for that one... Sorry.
%\input{app2}
%%%% Input bibliography file %%%%%%%%%%%%%%%
\begin{thebibliography}{99}\addcontentsline{toc}{chapter}{Bibliography}

%%BOVY: MOVING GROUPS I
\expandafter\ifx\csname natexlab\endcsname\relax\def\natexlab#1{#1}\fi

\bibitem[Aarseth \& Woolf(1972)]{Aarseth72a}
  Aarseth,~S.~J. \& Woolf,~N.~J., 1972,
  \aplett, 12, 159

\bibitem[Adams \etal(2001)]{Adams01a}
  Adams,~J.~D., Stauffer,~J.~R., Monet,~D.~G., Skrutskie,~M.~F., \& Beichman,~C.~A., 2001,
  \aj, 121, 2053

\bibitem[Adelberger, Heckel, \& Nelson(2003)]{adelberger}
  Adelberger,~E.~G., Heckel,~B.~R., \& Nelson,~A.~E., 2003,
%  Tests of the Gravitational Inverse-Square Law,
%  \textit{Annu.~Rev.~Nucl.~Part.~Sci.}, 53, 77
  Annu.~Rev.~Nucl.~Part.~Sci., 53, 77

\bibitem[Afflerbach, Churchwell, \& Werner(1997)]{Afflerbach97a}
  Afflerbach,~A., Churchwell,~E., \& Werner,~W.~M., 1997,
  \apj, 478, 190

\bibitem[{{Akaike}(1974)}]{Akaike}
{Akaike}, H. 1974, {IEEE Transactions on Automatic Control}, 19, 716

\bibitem[Allison \etal(2009)]{Allison09a}
  Allison,~R.~J., Goodwin,~S.~P., Parker,~R.~J., de Grijs,~R., Portegies Zwart,~S.~F., \& Kouwenhoven,~M.~B.~N., 2009,
  \apjl, 700, L99

\bibitem[Anderson \& Darling(1952)]{AndersonDarling}
  Anderson,~T.~W. \& Darling,~D.~A., 1952,
%  Asymptotic theory of certain ``goodness-of-fit'' criteria based on stochastic processes, 
%  \textit{Annals\,Math.\,Stat.}, 23, 193
  Ann.\,Math.\,Stat., 23, 193

\bibitem[{{Antoja} {et~al.}(2008){Antoja}, {Figueras}, {Fern{\'a}ndez}, \&
  {Torra}}]{2008A&A...490..135A}
{Antoja}, T., {Figueras}, F., {Fern{\'a}ndez}, D., \& {Torra}, J. 2008, \aap,
  490, 135

\bibitem[Antoja \etal(2009)]{Antoja09a}
  Antoja,~T., Valenzuela,~O., Pichardo,~B., Moreno,~E., Figueras,~F., \& Fern{\'a}ndez,~D., 2009,
  \apjl, 700, L78

\bibitem[Ascenso, Alves, \& Lago(2009)]{Ascenso09a}
  Ascenso,~J., Alves,~J., \& Lago,~M.~T.~V.~T., 2009,
  \aap, 495, 147

\bibitem[Asiain, Figueras, \& Torra(1999)]{Asiain99a}
  Asiain,~R., Figueras,~F., \& Torra,~J., 1999,
  \aap, 350, 434

\bibitem[{{Asiain} {et~al.}(1999){Asiain}, {Figueras}, {Torra}, \&
  {Chen}}]{1999A&A...341..427A}
{Asiain}, R., {Figueras}, F., {Torra}, J., \& {Chen}, B. 1999, \aap, 341, 427

\bibitem[Aumer \& Binney(2009)]{Aumer09a}
  Aumer,~M.~\& Binney,~J.~J., 2009,
  \mnras, 397, 1286

\bibitem[{{Baade}(1944)}]{1944ApJ...100..137B}
{Baade}, W. 1944, \apj, 100, 137

\bibitem[{{Baade}(1958)}]{baade1958a}
{Baade}, W. 1958, in {Stellar Populations: Proceedings of the conference
  sponsored by the Pontifical Academy of Science and the Vatican Observatory,
  May 20-28, 1957}, ed. D.~J.~K. {O'Connell} ({Amsterdam}: {North-Holland
  Pub.~Co.}), 3

\bibitem[{{Bahcall} \& {Soneira}(1980)}]{1980ApJS...44...73B}
{Bahcall}, J.~N. \& {Soneira}, R.~M. 1980, \apjs, 44, 73

\bibitem[{{Bahcall} \& {Soneira}(1984)}]{1984ApJS...55...67B}
{Bahcall}, J.~N. \& {Soneira}, R.~M. 1984, \apjs, 55, 67

\bibitem[Bailer-Jones(2008)]{bailerjones08a}
  Bailer-Jones, ~C.~A.~L., 2008,
  in IAU Symp.~254, The Galaxy Disk in Cosmological Context, ed.~J.~Andersen, J.~Bland-Hawthorn, \& B.~Nordstr\"{o}m, (Dordrecht: Kluwer), 475  

\bibitem[Beloborodov \& Levin(2004)]{roulette}
  Beloborodov,~A.~M. \& Levin,~Y., 2004,
%  Orbital roulette:\ A new method of gravity estimation from observed motions,
%  \textit{Astrophys.\,J.}, 613, 224
  \apj, 613, 224
\bibitem[Beloborodov \etal(2006)]{Beloborodov06a} 
  Beloborodov,~A.~M, Levin,~Y., Eisenhauer,~F., Genzel,~R., 
  Paumard,~T., Gillessen,~S.,
  \& Ott,~T., 2004,
%  Clockwise stellar disk and the dark mass in the Galactic Center,
%  \textit{Astrophys.\,J.}, 648, 405
  \apj, 648, 405

\bibitem[Belokurov \etal(2006)]{belokurovfield}
  Belokurov,~V., \etal, 2006,
%  The Field of Streams: Sagittarius and its siblings,
%  \textit{Astrophys.\,J.\,Lett.}, 642, 137
  \apjl, 642, L137
\bibitem[Belokurov \etal(2007)]{belokurov}
  Belokurov,~V., \etal, 2007,
%  Cats and dogs, hair and a hero: A quintet of New Milky Way companions,
%  \textit{Astrophys.\,J.}, 654, 897
  \apj, 654, 897

\bibitem[Benjamin \etal(2005)]{Benjamin05a}
  Benjamin,~R.~A., \etal, 2005, \apj, 630, 149

\bibitem[Bensby \etal(2007)]{Bensby07a}
  Bensby,~T., Oey,~M.~S., Feltzing,~S., \& Gustafsson,~B., 2007,
  \apjl, 655, L89

\bibitem[Bertelli \etal(1994)]{Bertelli94a}
  Bertelli,~G., Bressan,~A., Chiosi,~C., Fagotto,~F., \& Nasi, E., 1994,
  \aaps, 106, 275

\bibitem[{{Bienaym{\'e}}(1999)}]{1999A&A...341...86B}
{Bienaym{\'e}}, O. 1999, \aap, 341, 86

\bibitem[Binney, Gerhard, \& Spergel(1997)]{binney97a}
  Binney,~J.~J., Gerhard,~O., \& Spergel,~D.~N., 1997,
  \mnras, 288, 365

\bibitem[{{Binney} \& {Merrifield}(1998)}]{1998gaas.book.....B}
{Binney}, J. \& {Merrifield}, M. 1998, {Galactic Astronomy} (Princeton
  University Press)

\bibitem[{{Binney} {et~al.}(1997){Binney}, {Dehnen}, {Houk}, {Murray}, \&
  {Penston}}]{1997ESASP.402..473B}
{Binney}, J.~J., {Dehnen}, W., {Houk}, N., {Murray}, C.~A., \& {Penston}, M.~J.
  1997, in ESA Special Publication, Vol. 402, Hipparcos - Venice '97, 473

\bibitem[{{Binney} \& {Tremaine}(2008)}]{2008gady.book.....B}
{Binney}, J. \& {Tremaine}, S. 2008, {Galactic Dynamics: Second Edition}
  (Princeton University Press)

\bibitem[Bissantz \& Gerhard(2002)]{bissantz02a}
  Bissantz,~N.~\& Gerhard,~O., 2002,
  \mnras, 330, 591

\bibitem[{{Blaauw} {et~al.}(1960){Blaauw}, {Gum}, {Pawsey}, \&
  {Westerhout}}]{1960MNRAS.121..123B}
{Blaauw}, A., {Gum}, C.~S., {Pawsey}, J.~L., \& {Westerhout}, G. 1960, \mnras,
  121, 123

\bibitem[Blaauw(1970)]{Blaauw70a}
  Blaauw,~A., 1970, in The Spiral Structure of our Galaxy, Proceedings from 38th IAU Symposium, Vol.~38, ed.~W.~Becker \& I.~Kontopoulos, 199

\bibitem[Blitz \& Spergel(1991)]{Blitz91a}
  Blitz,~L.~\& Spergel,~D.~N., 1991, \apj, 379, 631

\bibitem[{{Boesgaard} \& {Budge}(1988)}]{1988ApJ...332..410B}
{Boesgaard}, A.~M. \& {Budge}, K.~G. 1988, \apj, 332, 410

\bibitem[Boesgaard \& Friel(1990)]{Boesgaard90a}
  Boesgaard,~A.~M. \& Friel,~E.~D., 1990,
  \apj, 351, 467

\bibitem[Bok(1934)]{Bok34a}
  Bok,~B.~J., 1934,
  Harvard Circ., 384, 1

\bibitem[Bok(1936)]{Bok36a}
  Bok,~B.~J., 1936,
  Observatory, 59, 76

\bibitem[Bok(1946)]{Bok46a}
  Bok,~B.~J., 1946,
  \mnras, 106, 61

\bibitem[{{Boss}(1911)}]{1911AJ.....27...33B}
{Boss}, B. 1911, \aj, 27, 33

\bibitem[{{Boss}(1908)}]{boss08a}
{Boss}, L. 1908, \aj, 26, 31

\bibitem[Bonnell \& Davies(1998)]{Bonnell98a}
  Bonnell,~I.~A. \& Davies,~M.~B., 1998,
  \mnras, 295, 691

\bibitem[Bouvier \etal(1998)]{Bouvier98a}
  Bouvier,~J., Stauffer,~J.~R., Martin,~E.~L., Barrado y Navascues,~D., Wallace,~B., \& Bejar,~V.~J.~S., 1998,
  \aap, 336, 490

\bibitem[Bouvier \etal(2008)]{Bouvier08a}
  Bouvier,~J., \etal, 2008,
  \aap, 481, 661

\bibitem[Bovy \& Hogg(2010)]{Bovy10a} Bovy,~J.~\& Hogg,~D.~W., 2010,
  \apj, 717, 617

%\bibitem[Bovy, Murray, \& Hogg(2010)]{Bovy10b} Bovy,~J., Murray, I., \& Hogg,~D.~W., 2010,
%  \apj, 711, 1157

\bibitem[Bovy \etal(2009a)]{Bovy09b}
  Bovy,~J., Hogg,~D.~W., \& Rix, H.-W., 2009a,
  \apj, 704, 1704

\bibitem[Bovy \etal(2009b)]{Bovyveldist}
  Bovy,~J., Hogg,~D.~W., \& Roweis,~S.~T., 2009b,
%  The velocity distribution of nearby stars from \emph{Hipparcos} data. I. The significance of the moving groups,
%  \textit{Astrophys.\,J.}, 700, 1794
  \apj, 700, 1794 (\bhr)

\bibitem[{{Bovy} {et~al.}(2009c){Bovy}, {Hogg}, \& {Roweis}}]{BovyXD}
{Bovy}, J., {Hogg}, D.~W., \& {Roweis}, S.~T. 2009c, {arXiv:0905.2979 [stat.ME]}

\bibitem[{{Boyle} \& {McClure}(1975)}]{1975PASP...87...17B}
{Boyle}, R.~J. \& {McClure}, R.~D. 1975, \pasp, 87, 17

\bibitem[{{Bozdogan}(1983)}]{Bozdogan}
{Bozdogan}, H. 1983, {Determining the number of component clusters in the
  standard multivariate normal mixture model using model-selection criteria},
  Tech. rep., {TR UIC/DQM/A83-1, Quantitative Methods Dept., University of
  Illinois}

\bibitem[{{Breger}(1968)}]{1968PASP...80..578B}
{Breger}, M. 1968, \pasp, 80, 578

\bibitem[{{Cabrera-Ca\~{n}o} \& {Alfaro}(1990)}]{1990A&A...235...94C}
{Cabrera-Ca\~{n}o}, J. \& {Alfaro}, E.~J. 1990, \aap, 235, 94

\bibitem[Caloi \etal(1999)]{Caloi99a}
  Caloi,~V., Cardini,~D., D'Antona,~F., Badiali,~M., Emanuele,~A., \& Mazzitelli,~I., 1999,
  \aap, 351, 925

\bibitem[Carlberg \& Sellwood(1985)]{Carlberg85a}
  Carlberg,~R.~G.~\& Sellwood,~J.~A., 1985,
  \apj, 292, 79

\bibitem[Carraro \etal(2007)]{Carraro07a}
  Carraro,~G., de La Fuente Marcos,~R., Villanova,~S., Moni Bidin,~C., de La Fuente Marcos,~C., Baumgardt,~H., \& Solivella,~G., 2007,
  %title = ``{Observational templates of star cluster disruption. The stellar group NGC 1901 in front of the Large Magellanic Cloud}'',
  \aap, 466, 931

\bibitem[Chakrabarty(2007)]{Chakrabarty07a}
  Chakrabarty,~D., 2007,
  \aap, 467, 145

\bibitem[{{Chaudhuri}(1940)}]{1940MNRAS.100..574C}
{Chaudhuri}, P.~C. 1940, \mnras, 100, 574

\bibitem[{{Chen} {et~al.}(1997){Chen}, {Asiain}, {Figueras}, \&
  {Torra}}]{1997A&A...318...29C}
{Chen}, B., {Asiain}, R., {Figueras}, F., \& {Torra}, J. 1997, \aap, 318, 29

\bibitem[{{Chereul} {et~al.}(1998){Chereul}, {Creze}, \&
  {Bienayme}}]{1998A&A...340..384C}
{Chereul}, E., {Creze}, M., \& {Bienayme}, O. 1998, \aap, 340, 384

\bibitem[{{Chereul} {et~al.}(1999){Chereul}, {Cr{\'e}z{\'e}}, \&
  {Bienaym{\'e}}}]{1999A&AS..135....5C}
{Chereul}, E., {Cr{\'e}z{\'e}}, M., \& {Bienaym{\'e}}, O. 1999, \aaps, 135, 5

\bibitem[Chereul \& Grenon(2001)]{Chereul01a}
  Chereul,~E. \& Grenon,~M., 2001, in Dynamics of Star Clusters and the
  Milky Way, ASP Conference Series, Vol. 228, ed.~S.~Deiters, B.~Fuchs,
  A.~Just, R.~Spurzem, \& R.~Wielen (San Francisco, CA: ASP), 398

\bibitem[Cole \& Weinberg(2002)]{cole02a}
  Cole,~A.~A.~\& Weinberg,~M.~D., 2002,
  \apjl, 574, L43

\bibitem[Contopoulos(1975)]{Contopoulos75a}
  Contopoulos,~G., 1975,
  \apj, 201, 566

\bibitem[Contopoulos \& Grosb\o l(1986)]{Contopoulos86a}
  Contopoulos,~G.~\& Grosb\o l,~P., 1986,
  \aap, 155, 11

\bibitem[Contopoulos \& Grosb\o l(1989)]{Contopoulos89a}
  Contopoulos,~G.~\& Grosb\o l,~P., 1989,
  A\&AR, 1, 261

\bibitem[{{Conway} \& {Sloane}(1992)}]{conway92a}
{Conway}, J.~H. \& {Sloane}, N. J.~A. 1992, {Sphere Packings, Lattices and
  Groups} (London: {Springer-Verlag})

\bibitem[{{Dehnen}(1998)}]{1998AJ....115.2384D}
{Dehnen}, W. 1998, \aj, 115, 2384

\bibitem[{{Dehnen} \& {Binney}(1998)}]{1998MNRAS.298..387D}
{Dehnen}, W. \& {Binney}, J.~J. 1998, \mnras, 298, 387

\bibitem[Dehnen(1999a)]{dehnen99b}
  Dehnen,~W., 1999a, \aj, 118, 1201

\bibitem[Dehnen(1999b)]{dehnen99c}
  Dehnen,~W., 1999b, \apj, 524, L35

\bibitem[Dehnen(2000)]{dehnen00a}
  Dehnen,~W., 2000, \aj, 119, 800

\bibitem[de la Fuente Marcos(1995)]{delafuentemarcos95a}
  de la Fuente Marcos,~R., 1995,
  \aap, 301, 407

\bibitem[{{Dempster} {et~al.}(1977){Dempster}, {Laird}, \&
  {Rubin}}]{Dempster1977}
{Dempster}, A.~P., {Laird}, N.~M., \& {Rubin}, D.~B. 1977, Journal of the Royal
  Statistical Society. Series B (Methodological), 39, 1

\bibitem[De Silva \etal(2007)]{DeSilva07a}
  De Silva,~G.~M., Freeman,~K.~C., Bland-Hawthorn,~J., Asplund,~M., \& Bessell,~M.~S., 2007,
  \aj, 133, 694

\bibitem[De Simone, Wu, \& Tremaine(2004)]{deSimone04a}
  De Simone,~R., Wu,~X., \& Tremaine,~S., 2004,
  \mnras, 350, 627

\bibitem[de Zeeuw \etal(1999)]{deZeeuw99a}
  de Zeeuw,~P.~T., Hoogerwerf,~R., de Bruijne,~J.~H.~J., Brown,~A.~G.~A., \& Blaauw,~A., 1999,
  \aj, 117, 354

\bibitem[Dobbie \etal(2002)]{Dobbie02a}
  Dobbie,~P.~D., Kenyon,~F., Jameson,~R.~F., Hodgkin,~S.~T., Hambly,~N.~C., \& Hawkins,~M.~R.~S., 2002,
  \mnras, 329, 543

\bibitem[{{Eddington}(1910)}]{1910MNRAS..71...43E}
{Eddington}, A.~S. 1910, \mnras, 71, 43

\bibitem[Eggen(1958)]{Eggen58a}
  Eggen,~O.~J., 1958,
  \mnras, 118, 65

\bibitem[{{Eggen}(1958)}]{1958MNRAS.118..154E}
{Eggen}, O.~J. 1958, \mnras, 118, 154

\bibitem[{{Eggen}(1959{\natexlab{a}})}]{1959Obs....79..182E}
{Eggen}, O.~J. 1959{\natexlab{a}}, The Observatory, 79, 182

\bibitem[{{Eggen}(1959{\natexlab{b}})}]{1959Obs....79...88E}
{Eggen}, O.~J. 1959{\natexlab{b}}, The Observatory, 79, 88

\bibitem[{{Eggen}(1964)}]{1964RGOB...84..111E}
{Eggen}, O.~J. 1964, Roy.~Greenwich Obs.~Bull., 84, 111

\bibitem[{{Eggen}(1965)}]{1965Obs....85..191E}
{Eggen}, O.~J. 1965, The Observatory, 85, 191

\bibitem[{{Eggen}(1969)}]{1969PASP...81..553E}
{Eggen}, O.~J. 1969, \pasp, 81, 553

\bibitem[{{Eggen}(1970)}]{1970PASP...82...99E}
{Eggen}, O.~J. 1970, \pasp, 82, 99

\bibitem[{{Eggen}(1971{\natexlab{a}})}]{1971PASP...83..271E}
{Eggen}, O.~J. 1971{\natexlab{a}}, \pasp, 83, 271

\bibitem[{{Eggen}(1971{\natexlab{b}})}]{1971PASP...83..251E}
{Eggen}, O.~J. 1971{\natexlab{b}}, \pasp, 83, 251

\bibitem[{{Eggen}(1978)}]{1978ApJ...222..203E}
{Eggen}, O.~J. 1978, \apj, 222, 203

\bibitem[{{Eggen}(1983)}]{1983AJ.....88..813E}
{Eggen}, O.~J. 1983, \aj, 88, 813

\bibitem[{{Eggen}(1986)}]{1986AJ.....92..910E}
{Eggen}, O.~J. 1986, \aj, 92, 910

\bibitem[Eggen(1993)]{Eggen93a}
  Eggen,~O.~J., 1993,
  \aj, 106, 1885

\bibitem[Eggen(1996)]{Eggen96a}
  Eggen,~O.~J., 1996,
  \aj, 112, 1595

\bibitem[{{Eggen} \& {Sandage}(1959)}]{1959MNRAS.119..255E}
{Eggen}, O.~J. \& {Sandage}, A.~R. 1959, \mnras, 119, 255

\bibitem[Eisenstein \etal(2005)]{Eisenstein05a}
  Eisenstein,~D.~J., \etal, 2005, \apj, 633, 560

\bibitem[Elmegreen \& Elmegreen(1982)]{Elmegreen82a}
  Elmegreen,~D.~M.~\& Elmegreen,~B.~G., 1983,
  \mnras, 201, 1021

\bibitem[Elmegreen \& Elmegreen(1983)]{Elmegreen83a}
  Elmegreen,~D.~M.~\& Elmegreen,~B.~G., 1983,
  \apj, 267, 31

\bibitem[{{ESA}(1997)}]{ESA97a}
{ESA}, 1997, {The \emph{Hipparcos} and \emph{Tycho} Catalogues} (ESA SP-1200; Noordwijk: ESA)

\bibitem[{{Famaey} {et~al.}(2005){Famaey}, {Jorissen}, {Luri}, {Mayor}, {Udry},
    {Dejonghe}, \& {Turon}}]{2005A&A...430..165F}
  {Famaey}, B., {Jorissen}, A., {Luri}, X., {Mayor}, M., {Udry}, S., {Dejonghe},
  H., \& {Turon}, C. 2005, \aap, 430, 165

\bibitem[Famaey \etal(2007)]{Famaey07a}
  Famaey,~B., Pont,~F., Luri,~X., Udry,~S., Mayor,~M., \& Jorissen,~A., 2007,
  \aap, 461, 957

\bibitem[Famaey, Siebert, \& Jorissen(2008)]{Famaey08a}
  Famaey,~B., Siebert,~A., \& Jorissen,~A., 2008,
  \aap, 483, 453

\bibitem[Feast \& Whitelock(1997)]{Feast97a} Feast,~M. \& Whitelock,~P., 1997,
  \mnras, 291, 683

\bibitem[Feltzing \& Holmberg(2000)]{Feltzing00a}
  Feltzing,~S.~\& Holmberg,~J., 2000,
  \aap, 357, 153
  
\bibitem[{{Figueras} {et~al.}(1997){Figueras}, {Gomez}, {Asiain}, {Chen},
  {Comeron}, {Grenier}, {Lebreton}, {Moreno}, {Sabas}, \&
  {Torra}}]{1997ESASP.402..519F}
{Figueras}, F., {Gomez}, A.~E., {Asiain}, R., {Chen}, B., {Comeron}, F.,
  {Grenier}, S., {Lebreton}, Y., {Moreno}, M., {Sabas}, V., \& {Torra}, J.
  1997, in ESA Special Publication, Vol. 402, Hipparcos - Venice '97, 519--524

\bibitem[Fischbach \& Talmadge(1999)]{fischbach}
  Fischbach,~E. \& Talmadge,~C.~L., 1999
  The Search for Non-Newtonian Gravity,
  (Springer-Verlag)

\bibitem[Flynn \etal(2006)]{Flynn06a}
  Flynn,~C., Holmberg,~J., Portinari,~L., Fuchs,~B., \& Jahrei{\ss},~H., 2006,
  \mnras, 372, 1149

\bibitem[{{Francis} \& {Anderson}(2009)}]{2009arXiv0901.3503F}
{Francis}, C. \& {Anderson}, E. 2009, arXiv:0901.3503 [astro-ph]

\bibitem[Freeman(1987)]{Freeman87a}
  Freeman,~K.~C., 1987, \araa, 25, 603

\bibitem[Freeman \& Bland-Hawthorn(2002)]{Freeman02a}
  Freeman,~K.~\& Bland-Hawthorn,~J., 2002,
  \araa, 40, 487

\bibitem[Freeman \etal(2010)]{Freeman10a}
  Freeman,~K., Bland-Hawthorn,~J., \& Barden,~S., 2010, 
  AAO Newsletter, 117, 9

\bibitem[Fux(2001)]{fux01a}
  Fux,~R., 2001,
  \aap, 373, 511

\bibitem[Gardner \& Flynn(2010)]{gardner10a}
  Gardner,~E.~\& Flynn,~C., 2010,
  \mnras, 405, 545

\bibitem[{{Gelman} {et~al.}(2000){Gelman}, {Carlin}, {Stern}, \&
  {Rubin}}]{Gelman00a}
{Gelman}, A., {Carlin}, J.~B., {Stern}, H.~S., \& {Rubin}, D.~B. 2000,
  {Bayesian Data Analysis} ({Chapman \& Hall/CRC})

\bibitem[Georgelin \& Georgelin(1976)]{Georgelin76a}
  Georgelin,~Y.~M. \& Georgelin,~Y.~P., 1976, \aap, 49, 57

\bibitem[{{Gerbaldi} {et~al.}(1989)}]{1989Msngr..56...12G}
{Gerbaldi}, M. {et~al.} 1989, ESO Messenger, 56, 12

\bibitem[Ghez \etal(2008)]{Ghez08a}
  Ghez,~A.~M., \etal, 2008,
  \apj, 689, 1044

\bibitem[{{Ghigna} {et~al.}(1998){Ghigna}, {Moore}, {Governato}, {Lake},
  {Quinn}, \& {Stadel}}]{ghigna98a}
{Ghigna}, S., {Moore}, B., {Governato}, F., {Lake}, G., {Quinn}, T., \&
  {Stadel}, J. 1998, \mnras, 300, 146

\bibitem[Gillessen \etal(2009)]{Gillessen09a}
  Gillessen,~S., Eisenhauer,~F., Trippe,~S., Alexander,~T., Genzel,~R., Martins,~F., \& Ott, T., 2009
  \apj, 692, 1075

\bibitem[{{Gilmore} \& {Reid}(1983)}]{1983MNRAS.202.1025G}
{Gilmore}, G. \& {Reid}, N. 1983, \mnras, 202, 1025

\bibitem[Giorgini \etal(1996)]{Giorgini96a}
  Giorgini,~J.~D., Yeomans,~D.~K., Chamberlin,~A.~B., Chodas,~P.~W., Jacobson,~R.~A., Keesey,~M.~S., Lieske,~J.~H., Ostro,~S.~J., Standish,~E.~M., \& Wimberly,~R.~N., 1996, 
%  JPL's On-Line Solar System Data Service,
%  \textit{Bull.\,Amer.\,Astron.\,Soc.}, 28, 1158
  \baas, 28, 1158

\bibitem[{{Gomez} {et~al.}(1990){Gomez}, {Delhaye}, {Grenier}, {Jaschek},
  {Arenou}, \& {Jaschek}}]{1990A&A...236...95G}
{Gomez}, A.~E., {Delhaye}, J., {Grenier}, S., {Jaschek}, C., {Arenou}, F., \&
  {Jaschek}, M. 1990, \aap, 236, 95

\bibitem[Gratton(2000)]{Gratton00a}
  Gratton,~R., 2000,
  in Stellar Clusters and Associations: Convection, Rotation, and Dynamos,
  ed.~R.~Pallavicini, G.~Micela, \& S.~Sciortino (San Francisco, CA: ASP),
  198, 225


\bibitem[{{Green}(1985)}]{Green85a}
{Green}, R.~M. 1985, {Spherical astronomy} (Cambridge University Press)

\bibitem[{{Grenier} {et~al.}(1985){Grenier}, {Gomez}, {Jaschek}, {Jaschek}, \&
  {Heck}}]{1985A&A...145..331G}
{Grenier}, S., {Gomez}, A.~E., {Jaschek}, C., {Jaschek}, M., \& {Heck}, A.
  1985, \aap, 145, 331

\bibitem[{{Gr{\"u}nwald}(2007)}]{Grunwaldbook}
{Gr{\"u}nwald}, P.~D. 2007, {The minimum description length principle}
  (Cambridge, Massachusetts: {MIT Press})

\bibitem[Hambly \etal(1999)]{Hambly99a}
  Hambly,~N.~C., Hodgkin,~S.~T., Cossburn,~M.~R., \& Jameson,~R.~F., 1999,
  \mnras, 303, 835

\bibitem[{{Helmi} {et~al.}(2003){Helmi}, {White}, \&
  {Springel}}]{2003MNRAS.339..834H}
{Helmi}, A., {White}, S.~D.~M., \& {Springel}, V. 2003, \mnras, 339, 834

\bibitem[{{Hertzsprung}(1909)}]{1909ApJ....30..135H}
{Hertzsprung}, E. 1909, \apj, 30, 135

\bibitem[Hillenbrand \& Hartmann(1998)]{Hillenbrand98a}
  Hillenbrand,~L.~A. \& Hartmann,~L.~W., 1998,
  \apj, 492, 540

\bibitem[H{\o}g \etal(2000a)]{hog00a}
  H{\o}g,~E., Fabricius,~C., Makarov,~V.~V., Urban,~S., Corbin,~T., Wycoff,~G., Bastian,~U., Schwekendiek,~P., \& Wicenec,~A., 2000a,
  \aap, 355, L27

\bibitem[H{\o}g \etal(2000b)]{hog00b}
  H{\o}g,~E., Fabricius,~C., Makarov,~V.~V., Bastian,~U., Schwekendiek,~P., Wicenec,~A., Urban,~S., Corbin,~T., \& Wycoff,~G., 2000b
    \aap, 357, 367

\bibitem[{{Hogg} {et~al.}(2005){Hogg}, {Blanton}, {Roweis}, \&
  {Johnston}}]{2005ApJ...629..268H}
{Hogg}, D.~W., {Blanton}, M.~R., {Roweis}, S.~T., \& {Johnston}, K.~V. 2005,
  \apj, 629, 268

\bibitem[Holmberg, Nordstr{\"o}m, \& Andersen(2007)]{Holmberg07a}
  Holmberg,~J., Nordstr\"{o}m,~B., \& Andersen,~J., 2007,
  \aap, 475, 519

\bibitem[Holmberg, Nordstr{\"o}m, \& Andersen(2009)]{Holmberg09a}
  Holmberg,~J., Nordstr\"{o}m,~B., \& Andersen,~J., 2009,
  \aap, 501, 941

\bibitem[{{Ivezi{\'c}} {et~al.}(2008)}]{2008ApJ...684..287I}
{Ivezi{\'c}}, {\v Z}. {et~al.} 2008, \apj, 684, 287

\bibitem[Jaynes(2003)]{jaynes}
  Jaynes,~E.~T., 2003,
  \textit{Probability theory: the logic of science} (Cambridge University Press)

\bibitem[Jeans(1915)]{Jeans15a}
  Jeans,~J.~H., 1915,
  \mnras, 76, 70

\bibitem[Jeans(1935)]{Jeans35a}
  Jeans,~J.~H., 1935,
  Observatory, 58, 108

\bibitem[Jeffreys(1939)]{Jeffreys39a}
  Jeffreys,~H., 1939,
  %\textit{Theory of Probability} (Oxford: Clarendon Press)
  Theory of Probability (Oxford: Clarendon Press)

\bibitem[{{Johnson} \& {Soderblom}(1987)}]{1987AJ.....93..864J}
{Johnson}, D.~R.~H. \& {Soderblom}, D.~R. 1987, \aj, 93, 864

\bibitem[{{Johnston}(1998)}]{johnston98a}
{Johnston}, K.~V. 1998, \apj, 495, 297

\bibitem[Johnston \etal(1999)]{Johnston99}
  Johnston,~K.~V., Zhao,~H., Spergel,~D.~N., \& Hernquist, L., 1999,
%  Tidal streams as probes of the Galactic potential,
%  \textit{Astrophys.\,J.\,Lett.}, 512, L109 
  \apjl, 512, L109 

\bibitem[de Jong \etal(2010)]{deJong10a}
  de Jong,~J.~T.~A., Yanny,~B., Rix,~H.-W., Dolphin,~A.~E., Martin,~N.~F., \& Beers,~T.~C., 2010, \apj, 714, 663

\bibitem[Juri\'{c} \etal(2008)]{Juric08a}
  Juri\'{c},~M., \etal, 2008, \apj, 673, 864

\bibitem[Kaasalainen \& Binney(1994)]{Binney94}
  Kaasalainen,~M. \& Binney,~J., 1994,
%  Construction of invariant tori and integrable Hamiltonians,
%  \textit{Phys.\,Rev.\,Lett.}, 73, 2377
  Phys.\,Rev.\,Lett., 73, 2377

\bibitem[Kalnajs(1991)]{Kalnajs91a}
  Kalnajs,~A., 1991, in Dynamics of Disc Galaxies, ed.~B.~Sundelius (G\"{o}teborg: Dept. Astron. Astrophys., G\"{o}teborg Univ.), 323

\bibitem[{{Kapteyn}(1905)}]{kapteyn05a}
{Kapteyn}, J.~C. 1905, {British Assoc.~Adv.~Sci.~Rep.}, Sec.~A, 257

\bibitem[{{Kapteyn}(1914)}]{1914ApJ....40...43K}
{Kapteyn}, J.~C. 1914, \apj, 40, 43

\bibitem[Katz \etal(2004)]{Katz04a}
  Katz,~D., \etal, 2004,
\mnras, 354, 1223

\bibitem[Kendall(1938)]{Kendall38a}
  Kendall,~M., 1938,
%  A new measure of rank correlation,
%  \textit{Biometrika}, 30, 81
  Biometrika, 30, 81

\bibitem[Kepler(1609)]{Kepler}
  Kepler,~J., 1609,
%  \textit{Astronomia Nova},
  Astronomia Nova,
  trans.\ W.~H.~Donahue, 1992
  (Cambridge University Press)

\bibitem[King \etal(2003)]{King03a}
  King,~J.~R., Villarreal,~A.~R., Soderblom,~D.~R., Gulliver,~A.~F., \& Adelman,~S.~J., 2003,
  %title = ``{Stellar Kinematic Groups. II. A Reexamination of the Membership, Activity, and Age of the Ursa Major Group}'',
  \aj, 125, 1980

\bibitem[Klypin, Zhao, \& Sommerville(2002)]{Klypin02a}
  Klypin,~A., Zhao,~H., \& Somerville,~R.~S., 2002,
  \apj, 573, 597

\bibitem[Klypin \etal(1999)]{Klypin99a}
  Klypin,~A., Kravtsov,~A.~V., Valenzuela,~O., \& Prada,~F., 1999, \apj, 522, 82

\bibitem[Kochanek(1996)]{Kochanek96a}
  Kochanek,~C.~S., 1996,
  \apj, 457, 228

\bibitem[Kolmogorov(1941)]{Kolmogorov41a}
  Kolmogorov,~A.~N., 1941,
%  Confidence limits for an unknown distribution function,
%  \textit{Ann.~Math.~Stat.}, 12, 461
  Ann.~Math.~Stat., 12, 461

\bibitem[{{Kolmogorov}(1965)}]{kolmogorov65a}
{Kolmogorov}, A.~N. 1965, {Problems.~Inform.~Transmission}, 1, 1

\bibitem[Komatsu \etal(2011)]{Komatsu11a}
  Komatsu,~E., \etal, 2011, \apjs, 192, 18

\bibitem[{{Koposov} {et~al.}(2008)}]{koposov}
{Koposov}, S.~E. {et~al.} 2008, \apj, 686, 279

\bibitem[{{Koposov} {et~al.}(2009){Koposov}, {Yoo}, {Rix}, {Weinberg},
  {Macci{\`o}}, \& {Escud{\'e}}}]{Koposov:2009ru}
{Koposov}, S.~E., {Yoo}, J., {Rix}, H.-W., {Weinberg}, D.~H., {Macci{\`o}},
  A.~V., \& {Escud{\'e}}, J.~M. 2009, \apj, 696, 2179

\bibitem[Koposov, Rix, \& Hogg(2009)]{Koposov09a} Koposov,~S.~E., Rix,~H.-W., \& Hogg,~D.~W., 2009, arXiv:0907.1085 [astro-ph]

\bibitem[Kuijken \& Gilmore(1989)]{Kuijken89a}
  Kuijken,~K.~\& Gilmore,~G., 1989, \mnras, 239, 605

\bibitem[Kuiper(1962)]{Kuiper62a}
  Kuiper,~N.~H., 1962,
%  Tests concerning random points on a circle,
%  \textit{Proc.~Koninklijke Nederlandse Akademie van Wetenschappen A}, 63, 38
  Proc.~Koninklijke Nederlandse Akademie van Wetenschappen A, 63, 38

\bibitem[Leonard \& Tremaine(1990)]{Leonard90a}
  Leonard,~P.~J.~T. \& Tremaine,~S., 1990,
  \apj, 353, 486
\bibitem[Leonard(1978)]{leonard1978}
  Leonard.,~T., 1978,
%  Density estimation, stochastic processes and prior information.
%  \textit{Journal of the Royal Statistical Society}, 40(2), 113
  J.\,Roy.\,Stat.\,Soc.\,B, 40, 113

\bibitem[Levine \etal(2006)]{Levine06a}
  Levine,~E.~S., Blitz,~L., \& Heiles,~C., 2006, \apj, 643, 881

\bibitem[Lin \& Shu(1964)]{Lin64a}
  Lin,~C.~C.~\& Shu,~F.~H., 1964,
  \apj, 140, 646

\bibitem[Little \& Tremaine(1987)]{Little87a}
  Little,~B.~\& Tremaine,~S., 1987,
%  Distant satellites as probes of our Galaxy's mass distribution
  \apj, 320, 493

\bibitem[Lovelace \& Hohlfeld(1978)]{Lovelace78a}
  Lovelace,~R.~V.~E.,~\& Hohlfeld,~R.~G., 1978,
  \apj, 221, 51

\bibitem[{{Luri} {et~al.}(1996){Luri}, {Mennessier}, {Torra}, \&
  {Figueras}}]{1996A&AS..117..405L}
{Luri}, X., {Mennessier}, M.~O., {Torra}, J., \& {Figueras}, F. 1996, \aaps,
  117, 405

\bibitem[Lynden-Bell \& Kalnajs(1972)]{LyndenBell72a}
  Lynden-Bell,~D. \& Kalnajs,~A.~J., 1972,
  \mnras, 157, 1

\bibitem[Mackay(2003)]{mackay}
  Mackay,~D.~J.~C., 2003,
  Information theory, inference, and learning algorithms (Cambridge University Press)

\bibitem[{{M{\"a}dler}(1846)}]{1846AN.....24..213M}
{M{\"a}dler}, J.~H. 1846, AN, 24, 213

\bibitem[{{M{\"a}dler}(1847)}]{madler47}
{M{\"a}dler}, J.~H. 1847, {Untersuchungen {\"u}ber die Fixstern-systeme}
  (Lepizig: Mitau)

\bibitem[Ma{\'{\i}}z Apell{\'a}niz(2006)]{Maiz06a}
  Ma{\'{\i}}z Apell{\'a}niz,~J., 2006,
    %title = ``{A Recalibration of Optical Photometry: Tycho-2, Str{\"o}mgren, and Johnson Systems}'',
  \aj, 131, 1184

\bibitem[Marigo \etal(2008)]{Marigo08a}
  Marigo,~P., Girardi,~L., Bressan,~A., Groenewegen,~M.~A.~T., Silva,~L., \& Granato,~G.~L., 2008,
  %title = ``{Evolution of asymptotic giant branch stars. II. Optical to far-infrared isochrones with improved TP-AGB models}'',
  \aap, 482, 883

\bibitem[Marshall \etal(2006)]{marshall06a}
  Marshall,~D.~J., \etal, 2006,
  \aap, 453, 635

\bibitem[Mayor(1976)]{Mayor76a}
  Mayor,~M., 1976,
  \aap, 48, 301

\bibitem[{{McDonald} \& {Hearnshaw}(1983)}]{1983MNRAS.204..841M}
{McDonald}, A.~R.~E. \& {Hearnshaw}, J.~B. 1983, \mnras, 204, 841

\bibitem[McMillan, Vesperini, \& Portegies Zwart(2007)]{McMillan07a}
  McMillan,~S.~L.~W., Vesperini,~E., \& Portegies Zwart,~S.~F., 2007,
  \apjl, 655, L45

\bibitem[Meidt, Rand, \& Merrifield(2009)]{Meidt09a}
  Meidt,~S.~E., Rand,~R.~J., \& Merrifield,~M.~R., 2009,
  \apj, 702, 277

\bibitem[Merrifield(1992)]{Merrifield92a}
  Merrifield,~M.~R., 1992, \aj, 103, 1552

\bibitem[Merrifield, Rand, \& Meidt(2006)]{Merrifield09a}
  Merrifield,~M.~R., Rand,~R.~J., \& Meidt,~S.~E., 2006,
  \mnras, 366, 17

\bibitem[Mestel(1963)]{Mestel63a}
  Mestel,~L., 1963,
  \mnras, 126, 553

\bibitem[Minchev \etal(2009)]{Minchev09a}
  Minchev,~I., Boily,~C., Siebert,~A., \& Bienayme, O., 2009,
  \mnras, 407, 2122

\bibitem[Minchev \& Famaey(2009)]{Minchev09b}
  Minchev,~I.~\& Famaey,~B., 2009,
  \apj, 722, 112

\bibitem[Moeckel \& Bonnell(2009)]{Moeckel09a}
  Moeckel,~N. \& Bonnell,~I.~A., 2009,
  \mnras, 396, 1864

\bibitem[Moore \etal(1999)]{Moore99a}
  Moore,~B., Ghigna,~S., Governato, ~F., Lake,~G., Quinn,~T., Stadel, ~J.,  \& Tozzi,~P., 1999, \apj, 524, 19

\bibitem[{{Murenzi}(1989)}]{1989wtfm.conf..239M}
{Murenzi}, R. 1989, in Wavelets. Time-Frequency Methods and Phase Space, ed.
  J.-M. {Combes}, A.~{Grossmann}, \& P.~{Tchamitchian}, 239--+

\bibitem[{{Navarro} {et~al.}(2004){Navarro}, {Helmi}, \&
  {Freeman}}]{2004ApJ...601L..43N}
{Navarro}, J.~F., {Helmi}, A., \& {Freeman}, K.~C. 2004, \apjl, 601, L43

\bibitem[Neal(1993)]{neal1993}
  Neal.,~R.~M., 1993,
  Probabilistic inference using {M}arkov chain {M}onte {C}arlo methods.
  Technical Report CRG-TR-93-1, Department of Computer Science,
  University of Toronto
\bibitem[Neal(1999)]{neal1999a}
  Neal.,~R.~M., 1999,
%  Regression and classification using {G}aussian process priors.
  in Bayesian Statistics 6,
  ed.\,J.~M. Bernardo \etal. (Oxford University Press), 475
\bibitem[Neal(2003)]{neal2003a}
  Neal.,~R.~M., 2003,
%  Slice sampling.
%  \textit{Annals of Statistics}, 31(3), 705
  Ann.\,Stat., 31, 705
\bibitem[Newton(1687)]{Newton}
  Newton,~I., 1687,
%  \textit{Philosophiae Naturalis Principia Mathematica},
  Philosophiae Naturalis Principia Mathematica,
  trans.\ F.~Cajori, 1934
  (University of California Press)

\bibitem[{{Nordstr{\"o}m} {et~al.}(2004){Nordstr{\"o}m}, {Mayor}, {Andersen},
  {Holmberg}, {Pont}, {J{\o}rgensen}, {Olsen}, {Udry}, \&
  {Mowlavi}}]{2004A&A...418..989N}
{Nordstr{\"o}m}, B., {Mayor}, M., {Andersen}, J., {Holmberg}, J., {Pont}, F.,
  {J{\o}rgensen}, B.~R., {Olsen}, E.~H., {Udry}, S., \& {Mowlavi}, N. 2004,
  \aap, 418, 989

\bibitem[{{Ogorodnikov} \& {Latyshev}(1968)}]{1968SvA....12..279O}
{Ogorodnikov}, K.~F. \& {Latyshev}, I.~N. 1968, Soviet Astronomy, 12, 279

\bibitem[{{Ogorodnikov} \& {Latyshev}(1970)}]{1970SvA....13..934O}
{Ogorodnikov}, K.~F. \& {Latyshev}, I.~N. 1970, Soviet Astronomy, 13, 934

\bibitem[{{Ohlsson}(1932)}]{Ohlsson32a}
{Ohlsson}, J. 1932, Ann.~Lund Obs., 6

\bibitem[{{Oliver} \& {Baxter}(1994)}]{oliver94a}
{Oliver}, J.~J. \& {Baxter}, R.~A. 1994, {MML and Bayesianism: Similarities and
  Differences}, Tech. Rep. Tech Report 206, Dept. of Computer Science, Monash
  University, Clayton, Vic. 3168, Australia

\bibitem[{{Oliver} {et~al.}(1996){Oliver}, {Baxter}, \& {Wallace}}]{Oliver96a}
{Oliver}, J.~J., {Baxter}, R.~A., \& {Wallace}, C.~S. 1996, in In Machine
  Learning: Proceedings of the Thirteenth International Conference (ICML 96
  (Morgan Kaufmann Publishers), 364

\bibitem[Oort(1932)]{Oort32}
  Oort,~J.~H., 1932,
%  The force exerted by the stellar system in the direction perpendicular to the Galactic plane and some related problems,
%  \textit{Bull.\,Astron.\,Inst.\,Netherlands}, 6, 249
  Bull.\,Astron.\,Inst.\,Netherlands, 6, 249

\bibitem[{{Oort}(1958)}]{oort58a}
{Oort}, J.~H. 1958, in {Stellar Populations: Proceedings of the conference
  sponsored by the Pontifical Academy of Science and the Vatican Observatory,
  May 20-28, 1957}, ed. D.~J.~K. {O'Connell} ({Amsterdam}: {North-Holland
  Pub.~Co.}), 515

\bibitem[{{Ormoneit} \& {Tresp}(1996)}]{Ormoneit1995}
{Ormoneit}, D. \& {Tresp}, V. 1996, in {Advances in Neural Information
  Processing Systems 8, NIPS, Denver,CO, November 27-30, 1995}, ed. D.~S.
  {Touretzky}, M.~{Mozer}, \& M.~E. {Hasselmo} (MIT Press)

\bibitem[Pavani \etal(2001)]{Pavani01a}
  Pavani,~D.~B., Bica,~E., Dutra,~C.~M., Dottori,~H., Santiago,~B.~X., Carranza,~G., \& D{\'{\i}}az,~R.~J., 2001,
  %title = ``{Open clusters or their remnants: B and V photometry of NGC 1901 and NGC 1252}'',
  \aap, 374, 554

\bibitem[Percival, Salaris, \& Groenewegen(2005)]{Percival05a}
  Percival,~S.~M., Salaris,~M., \& Groenewegen,~M.~A.~T., 2005,
  %title = ``{The distance to the Pleiades. Main sequence fitting in the near infrared}'',
  \aap, 429, 887

\bibitem[Perryman \etal(1998)]{Perryman98a}
  Perryman,~M.~A.~C., Brown,~A.~G.~A., Lebreton,~Y., Gomez,~A., Turon,~C., de Strobel,~G.~C., Mermilliod,~J.~C., Robichon,~N., Kovalevsky,~J., \& Crifo,~F., 1998,
%    title = ``{The Hyades: distance, structure, dynamics, and age}'',
  \aap, 331, 81

\bibitem[{{Perryman} {et~al.}(2001)}]{2001A&A...369..339P}
{Perryman}, M.~A.~C. {et~al.} 2001, \aap, 369, 339

\bibitem[{{Plummer}(1913)}]{1913MNRAS..73..492P}
{Plummer}, H.~C. 1913, \mnras, 73, 492

\bibitem[Press \etal(2007)]{Press07a}
  Press,~W.~H., Teukolsky,~S.~A, Vetterling,~W.~T., \& Flannery,~B.~P., 2007,
  Numerical Recipes: The Art of Scientific Computing, 3rd Edition (Cambridge University Press)

\bibitem[{{Proctor}(1869)}]{proctor69a}
{Proctor}, R.~A. 1869, Proc.~Roy.~Soc.~London, 18, 169

\bibitem[{{Proust} \& {Foy}(1988)}]{1988Ap&SS.145...61P}
{Proust}, D. \& {Foy}, R. 1988, \apss, 145, 61

\bibitem[Quillen(2003)]{Quillen03a}
  Quillen,~A.~C., 2003, 
  \aj, 125, 785

\bibitem[Quillen \& Minchev(2005)]{Quillen05a}
  Quillen,~A.~C.~\& Minchev,~I., 2005,
  \aj, 130, 576

\bibitem[Quillen \etal(2009)]{Quillen09a}
  Quillen,~A.~C., Minchev,~I., Bland-Hawthorn,~J., \& Haywood,~M., 2009, \mnras, 397, 1599

\bibitem[Raboud \etal(1998)]{Raboud98a}
  Raboud,~D., Grenon,~M., Martinet,~L., Fux,~R., \& Udry,~S., 1998,
  \aap, 335, L61

\bibitem[{{Rasmuson}(1921)}]{rasmuson21}
{Rasmuson}, N.~H. 1921, Med.~Lunds.~Obs., {Ser.~II}, {No.~26}

\bibitem[Rasmussen and Williams(2006)]{rasmussen2005a}
  Rasmussen,~C.~E. \& Williams.,~C.~K.~I., 2006
  Gaussian Processes for machine learning (Cambridge, MA: MIT Press)

\bibitem[Reid \& Brunthaler(2004)]{Reid04a}
  Reid,~M.~J. \& Brunthaler,~A., 2004,
  \apj, 616, 872

\bibitem[Reid \& Hawley(1999)]{Reid99a}
  Reid,~I.~N. \& Hawley,~S.~L., 1999,
  \aj, 117, 343

\bibitem[Reid(1992)]{Reid92a}
  Reid,~N., 1992,
  \mnras, 257, 257

\bibitem[Reid \etal(2009)]{Reid09a} Reid,~M.~J., \etal, 2009,
  \apj, 700, 137 (\reid)

\bibitem[Reyl\'{e} \etal(2009)]{Reyle09a}
  Reyl\'{e},~C., Marshall,~D.~J., Robin,~A.~C., \& Schultheis,~M., \aap, 495, 819

\bibitem[{{Rissanen}(1978)}]{rissanen78a}
{Rissanen}, J. 1978, {Automatica}, 14, 465

\bibitem[{{Roberts} {et~al.}(1998){Roberts}, {Husmeier}, {Rezek}, \&
  {Penny}}]{roberts98a}
{Roberts}, S.~J., {Husmeier}, D., {Rezek}, I., \& {Penny}, W. 1998, IEEE
  Transactions on Pattern Analysis and Machine Intelligence, 20, 1133

\bibitem[Robin \etal(2005)]{robin05a}
  Robin,~A.~C., \etal, 2005,
  \aap, 430, 129

\bibitem[{{Roman}(1949)}]{1949ApJ...110..205R}
{Roman}, N.~G. 1949, \apj, 110, 205

\bibitem[{{Roman}(1954)}]{1954AJ.....59..307R}
{Roman}, N.~G. 1954, \aj, 59, 307

\bibitem[Ro{\v s}kar \etal(2008)]{roskar08a}
  Ro{\v s}kar,~R., Debattista,~V.~P., Quinn,~T.~R., Stinson,~G.~S., \& Wadsley,~J., 2008,
  \apjl, 684, L79

\bibitem[Rudolph \etal(2006)]{Rudolph06a}
  Rudolph,~A.~L., Fich,~M., Bell,~G.~R., Norsen,~T., Simpson,~J.~P., Haas,~M.~R., Erickson,~E.~F., 2006,
  \apjs, 162, 346

\bibitem[{{Russell}(1912)}]{1912AJ.....27...96R}
{Russell}, H.~N. 1912, \aj, 27, 96

\bibitem[Samland \& Gerhard(2003)]{samland03a}
  Samland,~M.~\& Gerhard,~O.~E., 2003,
  \aap, 399, 961

\bibitem[Scho\"{e}nrich \& Binney(2009)]{schoenrich09a}
  Scho\"{e}nrich,~R.~\& Binney,~J.~J., 2009,
  \mnras, 396, 203

\bibitem[{{Schuster} {et~al.}(2006){Schuster}, {Moitinho}, {M{\'a}rquez},
  {Parrao}, \& {Covarrubias}}]{2006A&A...445..939S}
{Schuster}, W.~J., {Moitinho}, A., {M{\'a}rquez}, A., {Parrao}, L., \&
  {Covarrubias}, E. 2006, \aap, 445, 939

\bibitem[{{Schwarz}(1978)}]{schwarz78a}
{Schwarz}, G. 1978, {Ann.~Stat.}, 6, 461

\bibitem[{{Schwarzschild}(1907)}]{schwarzschild07a}
{Schwarzschild}, K. 1907, {Nachrichten von der K{\"o}niglichen Gesellschaft der
  Wissenschaften zu G{\"o}ttingen}, 5, {614}

\bibitem[{{Schwarzschild}(1958)}]{schwarzschild58a}
{Schwarzschild}, M. 1958, in {Stellar Populations: Proceedings of the
  conference sponsored by the Pontifical Academy of Science and the Vatican
  Observatory, May 20-28, 1957}, ed. D.~J.~K. {O'Connell} ({Amsterdam}:
  {North-Holland Pub.~Co.}), 207

\bibitem[Schwarzschild(1979)]{Schwarzschild79}
  Schwarzschild,~M.,
%  A numerical model for a triaxial stellar system in dynamical equilibrium,
%  \textit{Astrophys.\,J.}, 232, 236
  \apj, 232, 236

\bibitem[Sellwood(2000)]{Sellwood00a}
  Sellwood,~J.~A., 2000,
  Ap\&SS, 272, 31

\bibitem[Sellwood(2010)]{sellwood10a}
  Sellwood,~J.~A., 2010,
  \mnras, 409, 145

\bibitem[Sellwood \& Binney(2002)]{sellwood02a}
  Sellwood,~J.~A.~\& Binney,~J.~J., 2002,
  \mnras, 336, 785

\bibitem[Sellwood \& Carlberg(1984)]{Sellwood84a}
  Sellwood,~J.~A.~\& Carlberg,~R.~G., 1984,
  \apj, 282, 61

\bibitem[Sellwood \& Kahn(1991)]{Sellwood91a}
  Sellwood,~J.~A.~\& Kahn,~F.~D., 1991,
  \mnras, 250, 278

\bibitem[Sellwood \& Lin(1989)]{Sellwood89a}
  Sellwood,~J.~A.~\& Lin,~D.~N.~C., 1989,
  \mnras, 240, 991

\bibitem[Sellwood \& Sparke(1988)]{Sellwood88a}
  Sellwood,~J.~A. \& Sparke,~L.~S., 1988,
  \mnras, 231, 25

\bibitem[Shaver \etal(1983)]{Shaver83a}
  Shaver,~P.~A., McGee,~R.~X., Newton,~L.~M., Danks,~A.~C., \& Pottasch,~S.~R., 1983,
  \mnras, 204, 53

\bibitem[Shen \etal(2010)]{Shen10a}
  Shen,~J., \etal, 2010, \apj, 720, L72

\bibitem[Shu(1969)]{shu69a}
  Shu,~F.~H., 1969,
  \apj, 158, 505

\bibitem[Siebert \etal(2003)]{Siebert03a}
  Siebert,~A., Bienaym\'{e},~O., \& Soubiran,~C., 2003, \aap, 399, 531

\bibitem[{{Silverman}(1986)}]{Silverman86a}
{Silverman}, B.~W. 1986, {Density Estimation for Statistics and Data Analysis}
  ({Chapman and Hall})


\bibitem[{{Simon} \& {Geha}(2007)}]{2007ApJ...670..313S}
{Simon}, J.~D. \& {Geha}, M. 2007, \apj, 670, 313

\bibitem[Sirko \etal(2004)]{Sirko04a} Sirko,~E., Goodman,~J., Knapp,~J.~R., Brinkmann,~J., Ivezi\'{c},~Z., Knerr,~E.~J., Schlegel,~D.~J., Schneider,~D.~P., \& York,~D.~G., 2004,
  \aj, 127, 914

\bibitem[Sivia \& Skilling(2006)]{Sivia06a}
  Sivia,~D.~S.~\& Skilling,~J., 2006,
  Data Analysis:\ A Bayesian Tutorial (Oxford University Press)

\bibitem[{{Skuljan} {et~al.}(1999){Skuljan}, {Hearnshaw}, \&
  {Cottrell}}]{1999MNRAS.308..731S}
{Skuljan}, J., {Hearnshaw}, J.~B., \& {Cottrell}, P.~L. 1999, \mnras, 308, 731

\bibitem[{{Slezak} {et~al.}(1990){Slezak}, {Bijaoui}, \&
  {Mars}}]{1990A&A...227..301S}
{Slezak}, E., {Bijaoui}, A., \& {Mars}, G. 1990, \aap, 227, 301

\bibitem[{{Soderblom} \& {Clements}(1987)}]{1987AJ.....93..920S}
{Soderblom}, D.~R. \& {Clements}, S.~D. 1987, \aj, 93, 920

\bibitem[{{Soderblom} \& {Mayor}(1993)}]{1993AJ....105..226S}
{Soderblom}, D.~R. \& {Mayor}, M. 1993, \aj, 105, 226

\bibitem[{{Solomonoff}(1964{\natexlab{a}})}]{solomonoff64a}
{Solomonoff}, R. 1964{\natexlab{a}}, {Information and Control}, 7, 1

\bibitem[{{Solomonoff}(1964{\natexlab{b}})}]{solomonoff64b}
{Solomonoff}, R. 1964{\natexlab{b}}, {Information and Control}, 7, 224

\bibitem[{{Soubiran} {et~al.}(1990){Soubiran}, {Gomez}, {Arenou}, \&
  {Bougeard}}]{1990ebua.conf..407S}
{Soubiran}, C., {Gomez}, A.~E., {Arenou}, F., \& {Bougeard}, M.~L. 1990, in
  Errors, Bias and Uncertainties in Astronomy, ed. C.~{Jaschek} \&
  F.~{Murtagh}, 407--+

\bibitem[Smith \etal(2007)]{Smith07a}
  Smith,~M.C., \etal, 2007, \mnras, 379, 755

\bibitem[Standish(1998)]{Standish98a}
  Standish,~E.~M., 1998,
%  \textit{JPL Planetary and Lunar Ephemerides, DE405/LE405},
  JPL Planetary and Lunar Ephemerides, DE405/LE405,
  JPL IOM 312, F-98-048
\bibitem[Standish(2004)]{Standish04a}
  Standish,~E.~M., 2004,
%  An approximation to the errors in the planetary ephemerides of the Astronomical Almanac, 
%  \textit{Astron.~\& Astrophys.}, 417, 1165
  \aap, 417, 1165

\bibitem[{{Starck} \& {Murtagh}(2006)}]{Starck06}
{Starck}, J.-L. \& {Murtagh}, F. 2006, {Astronomical Image and Data Analysis}
  ({Springer})

\bibitem[Stephens(1970)]{Stephens70a}
  Stephens,~M.~A., 1970,
%  Use of Kolmogorov--Smirnov, Cram\'er--Von Mises and related statistics without extensive tables,
%  \textit{J.~Royal~Stat.~Soc.~B}, 32, 115
  J.~Royal~Stat.~Soc.~B, 32, 115

\bibitem[{{Stone}(1974)}]{stone74a}
{Stone}, M. 1974, Journal of the Royal Statistical Society. Series B
  (Methodological), 36, 111

\bibitem[Talmadge \etal(1988)]{talmadge}
  Talmadge,~C., Berthias,~J.-P., Hellings,~R.~W., \& Standish,~E.~M., 1988,
%  Model-independent constraints on possible modifications of Newtonian gravity,
%  \textit{Phys.~Rev.~Lett.}, 61, 1159
  Phys.~Rev.~Lett., 61, 1159

\bibitem[Tegmark \etal(2006)]{Tegmark06a}
  Tegmark,~M., \etal, 2006, \prd, 74, 123507

\bibitem[Terlevich(1987)]{Terlevich87a}
  Terlevich,~E., 1987,
  \mnras, 224, 193

\bibitem[{{Tollerud} {et~al.}(2008){Tollerud}, {Bullock}, {Strigari}, \&
  {Willman}}]{2008ApJ...688..277T}
{Tollerud}, E.~J., {Bullock}, J.~S., {Strigari}, L.~E., \& {Willman}, B. 2008,
  \apj, 688, 277

\bibitem[Toomre(1969)]{Toomre69a}
  Toomre,~A., 1969,
  \apj, 158, 899

\bibitem[{{Tremaine}(1999)}]{1999MNRAS.307..877T}
{Tremaine}, S. 1999, \mnras, 307, 877

\bibitem[{{Tuominen} \& {Vilhu}(1979)}]{1979LIACo..22..355T}
{Tuominen}, I.~V. \& {Vilhu}, O. 1979, in Liege International Astrophysical
  Colloquia, Vol.~22, Liege International Astrophysical Colloquia, 355--360

\bibitem[{{Ueda} {et~al.}(1998){Ueda}, {Nakano}, {Ghahramani}, \&
  {Hinton}}]{Naonori1998}
{Ueda}, N., {Nakano}, R., {Ghahramani}, Z., \& {Hinton}, G.~E. 1998, in {Neural
  Networks for Signal Processing VIII, 1998. Proceedings of the 1998 IEEE
  Signal Processing Society Workshop}, 274

\bibitem[{{van Leeuwen}(2007{\natexlab{a}})}]{2007ASSL..250.....V}
{van Leeuwen}, F. 2007{\natexlab{a}}, Astrophysics and Space Science Library,
  Vol. 250, {Hipparcos, the New Reduction of the Raw Data} ({Springer})

\bibitem[{{van Leeuwen}(2007{\natexlab{b}})}]{2007A&A...474..653V}
{van Leeuwen}, F. 2007{\natexlab{b}}, \aap, 474, 653

\bibitem[{{Wallace}(2004)}]{wallacebook}
{Wallace}, C.~S. 2004, {Statistical and inductive inference by minimum message
  length} ({Springer})

\bibitem[{{Wallace} \& {Boulton}(1968)}]{wallace68a}
{Wallace}, C.~S. \& {Boulton}, D.~M. 1968, {Computer Journal}, 11, 185

\bibitem[{{Wallace} \& {Dowe}(1999)}]{Wallace99a}
{Wallace}, C.~S. \& {Dowe}, D.~L. 1999, {The Computer Journal}, 42, 270

\bibitem[{{Wallace} \& {Freeman}(1987)}]{wallace87a}
{Wallace}, C.~S. \& {Freeman}, P.~R. 1987, Journal of the Royal Statistical
  Society. Series B (Methodological), 49, 240

\bibitem[Wasserman(2005)]{Wasserman05a}
  Wasserman,~L., 2006,
  All of Nonparametric Statistics (New York: Springer)

\bibitem[Weinberg(1994)]{Weinberg94a}
  Weinberg,~M.~D., 1994,
  \apj, 420, 597

\bibitem[Wielen(1971)]{Wielen71a}
  Wielen,~R., 1971, \aap, 13, 309

\bibitem[{{Williams} {et~al.}(2009){Williams}, {Freeman}, {Helmi}, \& {the RAVE
  collaboration}}]{2009IAUS..254..139W}
{Williams}, M.~E.~K., {Freeman}, K.~C., {Helmi}, A., \& {the RAVE
  collaboration}. 2009, in IAU Symposium, ed. J.~{Andersen},
  J.~{Bland-Hawthorn}, \& B.~{Nordstr{\"o}m}, Vol. 254, 139--144

\bibitem[Williams, Turyshev, \& Boggs(2004)]{williams}
  Williams,~J.~G., Turyshev,~S.~G., \& Boggs,~D.~H., 2004,
%  Progress in Lunar Laser Ranging Tests of Relativistic Gravity,
%  \textit{Phys.~Rev.~Lett.}, 93, 261101
  Phys.~Rev.~Lett., 93, 261101

\bibitem[{{Williams}(1971)}]{1971MNRAS.153..171W}
{Williams}, P.~M. 1971, \mnras, 153, 171

\bibitem[Willman \etal(2005)]{willman}
  Willman,~B., \etal, 2005,
%  A new Milky Way dwarf galaxy in Ursa Major,
%  \textit{Astrophys.\,J.\,Lett.}, 626, 85
  \apjl, 626, L85

\bibitem[{{Wilson}(1966)}]{1966Sci...151.1487W}
{Wilson}, O.~C. 1966, Science, 151, 1487

\bibitem[{{Wilson}(1932)}]{1938AJ.....47...49R}
{Wilson}, R.~E. 1932, \aj, 42, 49

\bibitem[{{Windham} \& {Cutler}(1992)}]{Windham}
{Windham}, M.~P. \& {Cutler}, A. 1992, {J.~Am.~Stat.~Assoc.}, 87, 1188

\bibitem[Xue \etal(2008)]{Xue08a} Xue,~X.~X., \etal, 2008, 
  \apj, 684, 1143

\bibitem[{{Zador}(1963)}]{zador63}
{Zador}, P.~L. 1963, PhD thesis, Stanford U.

\bibitem[{{Zador}(1982)}]{zador82}
{Zador}, P.~L. 1982, {IEEE~Trans.~Information theory}, 28, 139

\bibitem[{{Zhao} {et~al.}(2009){Zhao}, {Zhao}, \& {Chen}}]{2009ApJ...692L.113Z}
{Zhao}, J., {Zhao}, G., \& {Chen}, Y. 2009, \apjl, 692, L113


\end{thebibliography}


\end{document}
