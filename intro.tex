In the current understanding of the formation of structure in the
Universe, the initial tiny density fluctuations are generated during an
inflationary epoch. The dominant mass component---the dark
matter---starts to cluster under the influence of gravity after the
dark matter decouples from the other constituents of the early
Universe, while the ordinary matter of atoms and everyday life remains
coupled to radiation until about 300,000 years after the Big
Bang. When the temperature of the Universe drops below about 3,000 K,
ordinary matter decouples from radiation and falls into the potential
wells created by dark-matter clustering. Ordinary matter is
predominantly found as neutral hydrogen at this stage and this gas
settles at high density at the bottoms of the dark matter
potentials. Stars and later galaxies form in these high density regions.

The current paradigm has been successful in explaining the large-scale
properties of the Universe, especially the close-to-linear regime of
the Cosmic Microwave Background \citep{Komatsu11a}, formed at the
epoch of ordinary-matter--radiation decoupling, and the largest ($>
100$ Mpc) scales in the nearby
Universe \citep{Eisenstein05a,Tegmark06a}, where the non-linear
influence of gravity does not play a large role. However, the
formation and evolution of individual galaxies remains to be
understood. This is obvious from the many classes of galaxies and
other objects identified locally (\eg, the Hubble sequence, dwarf
Spheroidal galaxies, dwarf irregular galaxies, etc.)  and at higher
redshift (\eg, Ultraluminous infrared galaxies, sub-millimeter
galaxies, Lyman-limit systems, Damped Lyman-$\alpha$ absorbers, MgII
absorbers, etc.). How all these systems form and relate to one another
is yet to be explained.

Since most of Galactic and extra-galactic astronomy is limited to
snapshots at a particular time, one can study the formation and
evolution of galaxies either by looking at galaxies at different
redshifts and interpreting these observations as different stages in
the evolution of present-day galaxies, or one can study galaxies
locally, where observations with much more detail are possible,
especially for the Milky Way and M31, and infer their formation and
evolution. These two approaches move in remarkable lockstep, as the
highest redshifts studied by current and planned
surveys are comparable to the times at which the lowest metallicity
stars currently observed in the Milky Way
formed \citep{Freeman02a}. These approaches are therefore strongly
interrelated. This dissertation focuses on the latter approach.

The Milky Way is a unique cosmological laboratory. Because it seems
that the Milky Way is generic in many ways (see below), its formation
and evolution should also be generic and a detailed study of this one
galaxy sheds light on the formation of all (disk) galaxies. In
particular we note that the Milky Way's density profile contains much
information about the dark matter on sub-galactic scales, where
information is currently lacking, and that the formation and evolution
of the various Milky-Way components should mimick that of those
observed in external galaxies. The absence of sufficiently many
overdensities in the Milky Way's density profile (\ie, small
satellites orbiting the Galaxy) as compared to numerical
simulations \citep{Klypin99a,Moore99a} was considered anomalous in the
standard cosmological paradigm, although recent work has shown that
the discrepancy can be largely explained by detailed galaxy-formation
physics \citep[\eg,][]{Koposov:2009ru}. The fact that the Milky Way
has a bar is an important constraint on the formation of large spiral
galaxies \citep{Klypin02a}. These are just a few examples that show
that studies of the Milky Way can provide detailed constraints on the
$\Lambda$CDM cosmological model.

Our Galaxy is generic in many ways. It has most of the components disk
galaxies in the local Universe have: a thin disk, a thick
disk \citep{1983MNRAS.202.1025G}, a bulge and a bar \citep{Blitz91a},
a stellar halo \citep{Freeman87a} and a dark halo, and other features
such as spiral arms \citep{Georgelin76a} and a
warp \citep{Levine06a,Reyle09a}. None of these components or the
relations between them seem to be particularly different from similar
features observed in external galaxies: \eg, the thin disk is not
especially thin \citep{Juric08a}; the Milky Way lies only 1$\sigma$ from
the Tully--Fisher relation assuming a circular velocity of 220 km
$s^{-1}$ \citep{Flynn06a}. Further observations of both the Milky Way
and external galaxies are needed to confirm that the Milky Way is
generic (\eg, in the $K$-band Tully--Fisher relation, or whether the
Galaxy's circular velocity is not higher than 220 km s$^{-1}$,
see \chaptername~\ref{chap:masers}).

If the Milky Way is typical, we can use it to study how galaxies form
and evolve. Disentangling the formation of the Galaxy from its
subsequent evolution is difficult, so typically these need to be
studied concurrently using the same data. For example, the thick disk
is often considered to be a relic from the time that the Milky Way's
disk formed \citep{Freeman02a}---the thick disk forming early by
heating the proto-stellar disk---and its structure in
phase-space--age--metallicity dimensions should be informative about
disk formation. However, it has recently become clear that significant
radial mixing of stars in the disk can ``wash out'' the initial
structure of the disk---radial mixing happens easily in thin disks
through the scattering of stars at corotation of a spiral density wave
\citep{sellwood02a}, but accretion events can also induce radial
mixing \citep{Quillen09a}, such that it could affect the thick disk as
well. Similar evolutionary effects could be at play in the stellar
halo, making its interpretation more difficult in terms of the
formation of the Galaxy. In the thin disk radial mixing is presumably
very efficient---through spiral scattering or the combined effect of
the bar and spiral structure \citep{sellwood02a,Minchev09b}---such
that observations of thin disk stars mainly tells us about disk
evolution rather than disk formation.

The dark halo and objects orbiting within it still represents a
largely untapped reservoir of information about the Milky Way. Many of
the basic facts about the dark matter halo remain with large
uncertainties (often as large as 50 to 100\,percent): the extent of
the dark matter halo, the total mass of the Milky Way (current best
estimates put it at 1 to 2 times 10$^{12}$ \Msol); the scale radius of
the dark halo (estimates range from 10 to 30
kpc; \citealt{Smith07a,Xue08a}); the flattening of the halo; and the
inner slope of the density profile (estimates run from very cuspy
logarithmic slope of -2 to a flat, cored
profile; \citealt{2008gady.book.....B}). The basic shape and the mass
of the disk are also important when studying the dark halo and these
parameters also remain poorly known: estimates of the disk
scale-length run from 2 to 4 kpc, which leads to large uncertainties
in the mass of the disk. The time dependence of all of these
parameters, in particular of the total mass of the Milky Way, is also
largely unconstrained.

Much of the current observational constraints beyond the immediate
Solar neighborhood are purely photometric, or, equivalently,
static. The kinematics of the Solar neighborhood was studied in detail
by the \Hipparcos\ astrometric space mission \citep{ESA97a}, but this
was mainly limited to a sphere of 100 pc centered on the Sun and,
thus, only represent a tiny part of the Milky Way---depending on one's
interests this is either $\approx (0.1/100)^3 = 10^{-9}$ of the volume
of the Milky Way, or $\approx (0.1/10)^2 = 10^{-4}$ of the surface of
the disk; limited studies covering a larger fraction have been
performed (\eg, using BHB stars: \citealt{Xue08a}; using gas
kinematics \citealt{Merrifield92a}. Photometrically, the structure of
many components are well established, \eg, the vertical structure of
the thin and thick disks \citep{Juric08a}, the structure of stellar
halo \citep{deJong10a}; the bulge and bar \citep{Blitz91a}; some
components of the spiral structure of the
disk \citep{Benjamin05a}. However, \emph{dynamical} constraints are
mostly lacking: the disk scale-height is barely detected dynamically
\citep{Kuijken89a,Siebert03a}; there are hardly any
direct dynamical constraints on the disk scale-length (most
constraints from dynamical models on this quantity come from
combinations of various data sets, which can in most cases be fit with
a wide range of disk scale-lengths); the dynamical influence of the
bar and spiral structure is only just beginning to be
studied \citep[\eg,][]{dehnen00a,fux01a,deSimone04a,Quillen05a,Antoja09a}. In
a sense purely photometric observations can be a distraction, since
they often highlight dynamically irrelevant structures: examples of
this include spiral structure in the gaseous component or in the
distribution of young stars, which is not necessarily related to the
dynamically interesting density waves in the distribution of old
stars; observations of various bars in the central part of the Galaxy,
most of which are dynamically sub-dominant to the main bar and thus do
not influence the evolution of the disk
greatly \citep[\eg][]{Nishiyama05a,Benjamin05a,Lopez07a,Green11a}. Dynamical
studies are the only way to sort out what the major influences in the
evolution of the Galaxy are.

Upcoming observational campaigns have the potential to revolutionize
the dynamical study of the Milky Way. Foremost among these is
the \Gaia\ mission, an astrometric mission similar to the \Hipparcos\
mission but that is 1000 times more accurate and thus has the
potential to study that Milky Way
globally \citep{2001A&A...369..339P}. Other upcoming surveys such
as \sdss\emph{-III}'s Apache Point Observatory Galactic Evolution
Experiment (\apogee; \citealt{Eisenstein11a}) will be able to study
the disk in a global manner through infrared spectroscopy that is able
to peer through the dust clouds that pervade the disk of the Galaxy
and that make optical observations (such as those performed by \Gaia)
difficult. \apogee\ will also furnish detailed elemental abundances
for about 100,000 stars, which will allow chemo-dynamical models to be
constrained in detail for the first time.

The work presented in this dissertation is in many ways a preparation
of the data-analysis framework and tools for these upcoming
surveys. The first part of the dissertation (\chaptername
s~\ref{chap:solarsystem} and \ref{chap:masers}) focuses on one of the
major questions described in this Introduction: the detailed inference
of the Milky Way's density profile using the \Gaia\ data. The second
part (\chaptername s~\ref{chap:veldist} through \ref{chap:hercules})
is concerned with constraining the dynamical influence of
non-axisymmetric disk features---primarily the bar and spiral
structure---through detailed observations of the phase-space structure
of disk stars and their elemental-abundance distributions. In both
parts of the dissertation we apply the methods to currently available
data and we infer the Milky Way's circular velocity
in \chaptername~\ref{chap:masers} and some details about the Milky
Way's bar and spiral structure in \chaptername s~\ref{chap:veldist}
and \ref{chap:groups}.

All of the \chaptername s in this dissertation have been published in
the refereed literature and all but one were co-authored. In the
first \chaptername\ I was responsible for developing the basic
analysis method (in discussion with my co-authors), implementing, and
executing it. Iain Murray independently implemented the analysis
framework and confirmed my results; he was also responsible for
developing and implementing the extensions to the basic method
described in \sectionname~\ref{sec:altmethods}. I wrote most of the
paper, with some parts written by David W.~Hogg and a few paragraphs
by Iain Murray. In the second \chaptername\ I developed and
implemented the method and wrote the full paper (with comments from my
co-authors). In \chaptername~\ref{chap:veldist} I extended the
deconvolution method developed by third author Sam Roweis and
introduced in \citet{2005ApJ...629..268H} to be applicable to the
velocity-distribution deconvolution problem (these extensions are
described in \citealt{BovyXD}), I implemented the full method in C to
attain the execution speed necessary to explore a wide range of
complexity of the velocity distribution. I also did all of the further
analysis and wrote the paper. In \chaptername~\ref{chap:groups} I
developed the tests of the different hypotheses, carried them out, and
wrote the paper, all under the watchful eye of my co-author.
