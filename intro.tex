In the current understanding of the formation of structure in the
Universe, the initial tiny density fluctuations are generated during an
inflationary epoch. The dominant mass component---the dark
matter---starts to cluster under the influence of gravity after the
dark matter decouples from the other constituents of the early
Universe, while the ordinary matter of atoms and everyday life remains
coupled to radiation until about 300,000 years after the Big
Bang. When the temperature of the Universe drops below about 3,000 K,
ordinary matter decouples from radiation and falls into the potential
wells created by dark-matter clustering. Ordinary matter is
predominantly found as neutral hydrogen at this stage and this gas
settles at high density at the bottoms of the dark matter
potentials. Stars and later galaxies form in these high density regions.

The current paradigm has been successful in explaining the large-scale
properties of the Universe, especially the close-to-linear regime of
the Cosmic Microwave Background, formed at the epoch of
ordinary-matter--radiation decoupling, and the largest ($> 100$ Mpc)
scales in the nearby Universe, where the non-linear influence of
gravity does not play a large role. However, the formation and
evolution of individual galaxies remains to be understood. This is
obvious from the many classes of galaxies and other objects identified
locally (\eg, the Hubble sequence, dwarf Spheroidal galaxies, dwarf
irregular galaxies, etc.) and at higher redshift (\eg, Ultraluminous
infrared galaxies, sub-millimeter galaxies, Lyman-limit systems,
Damped Lyman-$\alpha$ absorbers, MgII absorbers, etc.). How all these
systems form and relate to one another is yet to be explained.

Since most of Galactic and extra-galactic astronomy is limited to
snapshots at a particular time, one can study the formation and
evolution of galaxies either by looking at galaxies at different
redshifts and interpreting these observations as different stages in
the evolution of present-day galaxies, or one can study galaxies
locally, where observations with much more detail are possible,
especially for the Milky Way and M31, and infer their formation and
evolution. These two approaches move in remarkable lockstep, as the
highest redshifts probed (BOVY: CHANGE) by current and planned surveys
are comparable to the times at which the lowest metallicity stars
currently studied in the Milky Way formed (BOVY: CITE
BLAND+FREEMAN). These approaches are therefore strongly
interrelated. This dissertation focuses on the latter approach.

The Milky Way is a unique cosmological laboratory. Because it seems
that the Milky Way is generic in many ways (SEE BELOW?), its formation
and evolution should also be generic and a detailed study of this of
this one galaxy sheds light on the formation of all (disk)
galaxies. In particular we note that the Milky Way's density profile
contains much information of the dark matter on sub-galactic scales,
where information is currently lacking, and that the formation and
evolution of the various Milky-Way components should mimick that of
those observed in external galaxies. The absence of sufficiently many
overdensities in the Milky Way's density profile (\ie, small
sattelites orbiting the Galaxy) as compared to numerical simulations
(BOVY: CITE MOORE AND KLYPIN, ETC.) was considered anomalous in the
standard cosmological paradigm, although recent work has shown that
the discrepancy can be largely explained by detailed galaxy-formation
physics (BOVY: CITE KOPOSOV ETC.). The fact that the Milky Way has a
bar is an important constraint on the formation of large spiral
galaxies (BOVY: CITE KLYPIN). These are just a few examples that show
that studies of the Milky Way can provide detailed constraints on the
$\Lambda$CDM cosmological model.

Our Galaxy is generic in many ways. It has most of the components disk
galaxies in the local Universe have: a thin disk, a thick disk, a
bulge and a bar, a stellar halo and a dark halo, and other features
such as spiral arms and a warp (BOVY: CITE SOME STUFF). None of these
components or the relations between them seem to be particularly
different from similar features observed in external galaxies: \eg,
the thin disk is not especially thin (BOVY: CITE); the Milky Way lies
only 1$\sigma$ from the Tully--Fisher relation assuming a circular
velocity of 220 km $s^{-1}$ \citep{Flynn06a}. Further observations of
both the Milky Way and external galaxies are needed to confirm that
the Milky Way is generic (\eg, in the $K$-band Tully--Fisher relation,
or whether the Galaxy's circular velocity is not higher than 200 km
s$^{-1}$, see \chaptername~\ref{chapter:masers}). 

If the Milky Way is typical, we can use it to study how galaxies form
and evolve. Disentangling the formation of the Galaxy from its
subsequent evolution is difficult, so typically these need to be
studied concurrently using the same data. For example, the thick disk
is often considered to be a relic from the time that the Milky Way's
disk formed (BOVY: CITE BAND-HAWTHORN)---the thick disk forming early
by heating the proto-stellar disk---and its structure in
phase-space--age--metallicity BOVY:SPACE? should be informative about
disk formation. However, it has recently become clear that significant
radial mixing of stars in the disk can ``wash out'' the initial
structure of the disk---radial mixing happens easily in thin disks
through the scattering of stars at corotation of a spiral density wave
(BOVY: CITE SELLWOOD), but accretion events can also induce radial
mixing, such that it could affect the thick disk as well (BOVY: LOOK
AT BIRD PAPER AND FIND REFS). Similar evoutionary effects could be at
play in the stellar halo, making its interpretation more difficult in
terms of the formation of the Galaxy. In the thin disk radial mixing
is presumably very efficient---through spiral scattering or the
combined effect of the bar and spiral structure (BOVY: CITE SELLWOOD
AND ALSO MINCHEV)---such that observations of thin disk stars mainly
tells us about disk evolution rather than disk formation.

The dark 



Absence of major merger for MW (10 Gyr, redshift ?). Importance of
secular evolution in the interpretation of ``old'' stars, \eg, thick
disk as remnant of formation epoch.


It is known for certain that there is a large amount of dark mass
beyond the Sun’s position in the Milky Way. We expect there to be dark
matter in the inner region of the Galaxy as well. The determination of
the mass distribution in the inner Galaxy, and its decomposition into
the stellar and dark components, is made difficult by the ambiguity of
local stars as tracers of the mass distribution; certain distracting
disk dynamical features such as spiral arms; and our own (moving)
position in the middle of all of this.

While the Milky Way is interesting in its own right—--it is our home
in the Universe--—it is only one among thousands of galaxies in the
local Universe. Despite this, the Milky Way is one of the main sites
for studying structure formation in the Universe on small scales,
because of the amazing amount of detail possible in these
observations. For comparison, the apparent size of the Milky Way’s
nearest neighbor, the Andromeda galaxy, is only a few times that of
the full moon; other nearby galaxies are much farther away. Since we
have no reason to believe that the Milky Way is atypical in any
way—--we can detect many galaxies with similar features in the night
sky—the details of the Milky Way’s mass distribution and formation
history also provide a general understanding of how galaxies form.


Since stars move across the sky at a slow pace and because they orbit
the Galaxy on timescales measured in hundreds of millions of years, we
can hope at best to measure their positions and velocities at the
present time, while their accelerations are beyond our observational
reach. This poses a fundamental difficulty for interpreting the data
in terms of the mass distribution that guides them.  Newton’s second
law says that the acceleration of a star is directly set by the force
acting on it, while its velocity merely reflects the initial motion of
the object and not the underlying force. Thus, without the
accelerations of stars it seems that we cannot know the underlying
mass distribution. This problem and its resolution is discussed in
Chapter I.

To derive the mass distribution of the Milky Way from stellar motions,
we need to make additional assumptions about the kinematics of stars
that couple these motions to the mass distribution. One such
assumption is that the distribution of the stellar positions and
velocities is time-independent.  This seemingly simple assumption
provides the coupling between the underlying mass distribution and the
stellar kinematics. As an example, I show that we can determine the
force law in the Solar System from observations of the planets’
positions and velocities at one epoch. The nontrivial part of this toy
data set’s analysis is that one has to model the distribution of the
planets’ positions and velocities in a timeindependent manner while
not introducing extra assumptions. I achieve this by building a full
probabilistic model of the data that introduces extra degrees of
freedom describing the time-independent distribution of the positions
and velocities; these extra degrees of freedom can be integrated over
to obtain the final result for the force law in the Solar System which
is independent of the extra degrees of freedom.

Another application considers not observations of stars directly, but
the light of young stars, reprocessed and amplified by the molecular
gas clouds surrounding them. These masers—--lasers on a cosmic
scale—--can be seen to large distances, and their kinematics can be
measured by radio observatories much more precisely than is possible
for stars with optical telescopes. The current data set of 18 masers
contains much information about the disk’s structure, but the
interpretation of the data again depends on our assumptions about the
distribution of their positions and velocities. By inferring a simple
model in which the velocities of the masers are scattered around a
mean offset from the velocity associated with circular motion, we
obtain this circular velocity at the Sun’s position in the
Galaxy. This tells us directly about the Galaxy’s mass within the
Solar radius. This work has been published (Bovy et al. 2009,
Astrophys. J.  704, 1704). 

Chapter III concerns the available data on the motions of stars in the
Galactic disk. The Hipparcos satellite has accurately measured the
positions and motions of stars close to the Sun, which allows us to
look at their detailed properties. First I reconstruct the local
velocity distribution from these data. The result is complex which one
would not expect if the assumptions—axisymmetry and timeindependence
—introduced in the previous paragraphs were correct (this work has
been published: Bovy et al. 2009, Astrophys. J. 700, 1794). Looking at
the mixture of stars that make up the unexpected complexity, I find
that these stars have not formed together such that the complexity is
not due to a formation-history time-dependence and axisymmetry must be
broken. By considering how consistently we can measure the Sun’s
motion relative to the local circular velocity (measured in chapter
II)—--Assuming axisymmetry, this measurement is possible—--I find that
the disk is non-axisymmetric at the level of a few percent.
