It is known for certain that there is a large amount of dark mass
beyond the Sun’s position in the Milky Way. We expect there to be dark
matter in the inner region of the Galaxy as well. The determination of
the mass distribution in the inner Galaxy, and its decomposition into
the stellar and dark components, is made difficult by the ambiguity of
local stars as tracers of the mass distribution; certain distracting
disk dynamical features such as spiral arms; and our own (moving)
position in the middle of all of this. 

While the Milky Way is interesting in its own right—--it is our home
in the Universe--—it is only one among thousands of galaxies in the
local Universe. Despite this, the Milky Way is one of the main sites
for studying structure formation in the Universe on small scales,
because of the amazing amount of detail possible in these
observations. For comparison, the apparent size of the Milky Way’s
nearest neighbor, the Andromeda galaxy, is only a few times that of
the full moon; other nearby galaxies are much farther away. Since we
have no reason to believe that the Milky Way is atypical in any
way—--we can detect many galaxies with similar features in the night
sky—the details of the Milky Way’s mass distribution and formation
history also provide a general understanding of how galaxies form.


Since stars move across the sky at a slow pace and because they orbit
the Galaxy on timescales measured in hundreds of millions of years, we
can hope at best to measure their positions and velocities at the
present time, while their accelerations are beyond our observational
reach. This poses a fundamental difficulty for interpreting the data
in terms of the mass distribution that guides them.  Newton’s second
law says that the acceleration of a star is directly set by the force
acting on it, while its velocity merely reflects the initial motion of
the object and not the underlying force. Thus, without the
accelerations of stars it seems that we cannot know the underlying
mass distribution. This problem and its resolution is discussed in
Chapter I.

To derive the mass distribution of the Milky Way from stellar motions,
we need to make additional assumptions about the kinematics of stars
that couple these motions to the mass distribution. One such
assumption is that the distribution of the stellar positions and
velocities is time-independent.  This seemingly simple assumption
provides the coupling between the underlying mass distribution and the
stellar kinematics. As an example, I show that we can determine the
force law in the Solar System from observations of the planets’
positions and velocities at one epoch. The nontrivial part of this toy
data set’s analysis is that one has to model the distribution of the
planets’ positions and velocities in a timeindependent manner while
not introducing extra assumptions. I achieve this by building a full
probabilistic model of the data that introduces extra degrees of
freedom describing the time-independent distribution of the positions
and velocities; these extra degrees of freedom can be integrated over
to obtain the final result for the force law in the Solar System which
is independent of the extra degrees of freedom.

Another application considers not observations of stars directly, but
the light of young stars, reprocessed and amplified by the molecular
gas clouds surrounding them. These masers—--lasers on a cosmic
scale—--can be seen to large distances, and their kinematics can be
measured by radio observatories much more precisely than is possible
for stars with optical telescopes. The current data set of 18 masers
contains much information about the disk’s structure, but the
interpretation of the data again depends on our assumptions about the
distribution of their positions and velocities. By inferring a simple
model in which the velocities of the masers are scattered around a
mean offset from the velocity associated with circular motion, we
obtain this circular velocity at the Sun’s position in the
Galaxy. This tells us directly about the Galaxy’s mass within the
Solar radius. This work has been published (Bovy et al. 2009,
Astrophys. J.  704, 1704). 

Chapter III concerns the available data on the motions of stars in the
Galactic disk. The Hipparcos satellite has accurately measured the
positions and motions of stars close to the Sun, which allows us to
look at their detailed properties. First I reconstruct the local
velocity distribution from these data. The result is complex which one
would not expect if the assumptions—axisymmetry and timeindependence
—introduced in the previous paragraphs were correct (this work has
been published: Bovy et al. 2009, Astrophys. J. 700, 1794). Looking at
the mixture of stars that make up the unexpected complexity, I find
that these stars have not formed together such that the complexity is
not due to a formation-history time-dependence and axisymmetry must be
broken. By considering how consistently we can measure the Sun’s
motion relative to the local circular velocity (measured in chapter
II)—--Assuming axisymmetry, this measurement is possible—--I find that
the disk is non-axisymmetric at the level of a few percent.
